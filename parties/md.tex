\subsubsection*{Histoire}

\paragraph{} Le Markdown est un \textbf{langage de balisage} crée en 2004. Il
est facile à manipuler, donc simple à écrire et à lire sans connaître les
balises. Le Markdown peut être écrit sur n'importe quel éditeur de texte, il
suffit, lorsque le document est prêt a être enregistré de nommer le document
avec l'extension : "\textbf{.md}".

\subsubsection*{Instructions}

\paragraph{} Les instructions sont \textbf{très simples et peuvent être
combinées}. On va voir quelques instructions de base:

\begin{minted}{md}
# Titre

## Sous-titre

### Sous-sous-titre (etc.)

Ceci est un paragraphe.
ceci fait partie du même paragraphe.

Ceci est un autre paragraphe.

- liste
- à
- puces

1. liste
2. numérotée
3. etc.

Il est possible de mettre le texte *en italique*, **en gras**,
_souligné_ ou encore ~~barré~~.

```
ceci est un extrait de code
```

```python
# Ceci est un extrait de code en Python
answer = 42
print("Hello, World: " + str(answer))
```

> Ceci est une citation
> sur plusieurs
> lignes

Les liens sont détectés automatiquement: https://ddg.gg/
[https://www.wikipedia.org/](Lien avec titre)

![https://placekitten.com/400/400](Image avec description)
\end{minted}
