\subsubsection{Histoire}

Le markdown est un \textbf{langage de balisage} crée en 2004. Il est facile à manipuler, donc simple à écrire et a lire sans connaître les balises.
Le markdown peut être écrit sur n'importe quel éditeur de texte, il suffit lorsque le document est prêt a être enregistrer de nommer le document puis d'inscrire "\textbf{.md}".


\subsubsection{Instruction}

Les instructions sont \textbf{très simple et peuvent être combiné}. 
On va voir quelques instructions de bases : 

\paragraph{Police d'un texte}
\begin{itemize}
	\item Italique : encadrée le(s) mot(s) désiré(s) par * ou \_
	\item Gras :  le(s) mot(s) désiré(s) par **
	\item Souligner : encadrée le(s) mot(s) désiré(s) par \_\_
	\item Barré : encadrée le(s) mot(s) désiré(s) par \~{}\~{}
	\item souligner les titres : mettre sur une ligne en dessous des = ou -
\end{itemize}


\paragraph{Mise en forme du texte}
\begin{itemize}
	\item commencer un paragraphe : mettre 4 espaces
	\item délimiter un paragraphe : sauter une ligne
	\item retour a la ligne : deux espaces a la fin de phrase
	\item titre : \# (rajouter des \# par sous niveaux de paragraphes)
\end{itemize}


\paragraph{Ajout d'élément au texte}
\begin{itemize}
	\item bloc de code : encadrée le(s) mot(s) désiré(s) par ```
	\item citation : commercer par \textgreater
	\item liste : commencer par * ou - ou +
	\item liste ordonnée : commencer par 1. 2. 3. ...
	\item cases a cocher : [ ] ou [x]
	\item tableau : 
		- delimité les colonnes par : \textbar
		- delimité les titres des autres lignes par : :-----:
	\item lien : 
		- en hypertexte : l'encadré en \textless et \textgreater
		- en bouton : [ \textit{nom du bouton} ](\#){.btn }
	\item mage : ![ \textit{texte} ]( \textit{url de l'image} )
\end{itemize}
