\textbf{F-Droid} est un magasin d'application (\textit{au même titre que le PlayStore ou l'AppleStore}) qui met à disposition plus de \textbf{12000 applications libres et gratuites} sur les mobiles Android.
Il a été crée en 2010 et est promu par la Free SoftWare Foundation Europe. Son architecture de sécurité est basé sur le model de Debian.
Comme toute construction OpenSource, F-Droid est tenu et développé par un grand nombre de contributeur faisant partie de la communauté. 
Ainsi chaque personne peut crée sa propre application et la mettre a disposition sur ce magasin.

L’avantage à utiliser F-Droid plutôt que le magasin d'application par défaut, est principalement la préservation de vos données. Mais aussi la \textbf{sécurité} de vos applications et le faite qu'il n'est pas nécessaire de s'identifier pour pouvoir télécharger des nouvelles applications. 
Ce magasin facilite la découverte et l'installation de multiple applications. De plus contrairement à la plupart des applications, vous n’êtes pas obliger de faire les mise à jour, et ainsi garder une ancienne version. 
