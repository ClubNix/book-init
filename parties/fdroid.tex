\paragraph{} \textbf{F-Droid} est un magasin d'application (au même titre que
le PlayStore ou l'AppleStore) pour smartphones Android qui met à disposition
plus de \textbf{12 000 applications libres et gratuites}.

\paragraph{} Il a été créé en 2010 et est promu par la Free Software Foundation
Europe. Son architecture de sécurité est basée sur le modèle de Debian. Comme
toute construction Open Source, F-Droid est tenu et développé par un grand
nombre de contributeurs faisant partie de la communauté. Ainsi, chaque personne
peut créer sa propre application et la mettre à disposition gratuitement sur ce
magasin. F-Droid assure un certain niveau de sécurité: les applications
publiées sont vérifiées.

\paragraph{} L’avantage d'utiliser F-Droid plutôt que le magasin d'application
par défaut est principalement la préservation de vos données.  Mais aussi la
\textbf{sécurité} de vos applications et le fait qu'il n'est pas nécessaire de
s'identifier pour pouvoir télécharger des nouvelles applications.

\paragraph{} Pour un développeur d'application Android, un avantage est aussi
le fait que mettre à disposition son application sur F-Droid est gratuit,
contrairement au PlayStore et à l'AppleStore.

\paragraph{} Ce magasin facilite la découverte et l'installation de multiples
applications. De plus, contrairement à la plupart des applications, vous
n’êtes pas obligé de faire les mises à jour, et pouvez ainsi garder une
ancienne version.
