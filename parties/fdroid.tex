\textbf{F-Droid} est un magasin d'application (\textit{au même titre que le PlayStore ou l'AppleStore}) qui met à disposition plus
de \textbf{12000 applications libres et gratuites} sur les mobiles Android.
Il a été créé en 2010 et est promu par la Free SoftWare Foundation Europe. Son architecture de sécurité est basée sur le modèle de Debian.
Comme toute construction OpenSource, F-Droid est tenu et développé par un grand nombre de contributeurs faisant partie de la communauté.
Ainsi, chaque personne peut créer sa propre application et la mettre à disposition sur ce magasin.

L’avantage à utiliser F-Droid plutôt que le magasin d'application par défaut est principalement la préservation de vos données.
Mais aussi la \textbf{sécurité} de vos applications et le fait qu'il n'est pas nécessaire de s'identifier pour pouvoir télécharger
 des nouvelles applications.
Ce magasin facilite la découverte et l'installation de multiples applications. De plus, contrairement à la plupart des applications,
 vous n’êtes pas obligé de faire les mises à jour, et pouvez ainsi garder une ancienne version. 
