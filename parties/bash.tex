\subsubsection{Bash et scripts}

Le bash est un interpréteur de script natif aux systèmes d'exploitation sous
environnement Unix.

\paragraph{A quoi ça sert ?}

Le bash permet l'exécution de scripts, petits morceaux de code, souvent exécutés
 au démarrage, afin d'effectuer une tâche précise.
Par exemple, l'ensemble des applications s'ouvrant au démarrage sont regroupées
 dans un script. Bash permet l'interprétation des commandes shell.
Une commande shell est une chaîne de caractères en minuscules qui peut être
invoquée au travers d'un invite de commande ou d'un script.
 Des options et des arguments peuvent la compléter.

Le bash propose différents traitements des commandes :
\begin{itemize}
  \item Les commandes simultanées : com1 \& com2 \& com3 \& ... \& comN. Les commandes com1 à comN sont exécutées simultanéments;
  \item Les commandes successives : com1 ; com2 ; com3 ; ... ; comN. Les commandes com 1 à comN sont exécutées les unes après les autres;
  \item etc,...\newline
\end{itemize}

Il existe deux modes de fonctionnement du bash :
\begin{itemize}
  \item Le mode interactif : Le terminal fonctionne avec le bash. lorsque l'on
  tape une commande manuellement dans le terminal, on crée en réalité un mini
  script d'une ligne, interprété par bash puis executé;
  \item Le mode batch : bash exécute automatiquement un script contenu dans un
  fichier texte contenant les commandes à utiliser. Ce fichier doit être
  exécutable (réglable avec chmod).
\end{itemize}


\paragraph{L'interprétation d'une ligne de commande}

Chaque interprétation d'une ligne de commande d'un script repose sur des codes bien stricts :
\begin{itemize}
  \item Le premier mot de la ligne est interprété comme le nom de la commande;
  \item Chaque mot est séparé par un ou plusieurs caractères de séparation (espace, tabulation, tiret, underscore ...);
  \item La fin de la ligne se termine par un ';', comme dans la plupart des langages de programmation, ou un retour à la ligne
\end{itemize}


Pour plus d'informations sur les lignes de commandes bash, il existe un poly sur les commandes du terminal disponible sur le GitHub du club.
