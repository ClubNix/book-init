\subsubsection{Bash et scripts}

\paragraph{} Le bash est un interpréteur de ligne de commande natif aux
systèmes d'exploitation UNIX et GNU/Linux.

\paragraph{A quoi ça sert?}

\paragraph{} Bash est le programme par défaut sous GNU/Linux qui exécute des
commandes. Il peut être utilisé en mode interactif, comme vu à la partie sur la
ligne de commande; mais il peut aussi être utilisé pour exécuter des scripts.

\paragraph{} Un script est tout simplement un regroupement de lignes de
commandes, qui vont être exécutées lignes par lignes.

\paragraph{L'interprétation d'une ligne de commande}

Chaque interprétation d'une ligne de commande, que ce soit en mode intéractif
ou dans un script, repose sur ce format:

\begin{itemize}
	\item Le premier mot de la ligne est interprété comme le nom de la commande
	\item Chaque mot est séparé par un ou plusieurs caractères de séparation
		(espace, tabulation)
	\item Un retour à la ligne, ou un ``\texttt{;}'' si l'on veut mettre
		plusieurs commandes sur une seule ligne
\end{itemize}

Bash propose différents traitements des commandes:
\begin{description}
	\item[Commandes successives:] \mintinline{bash}{com1 ; com2 ; ... ; comN.}
		Les commandes com1 à comN sont exécutées les unes après les autres
	\item[Commandes simultanées:] \mintinline{bash}{com1 & com2 & ...  & comN}.
		Les commandes com1 à comN sont exécutées simultanéments
\end{description}

\paragraph{} Voici un exemple de script Bash:

\begin{minted}{bash}
#!/bin/bash

echo "Enter username"
read username
echo "Enter password"
read password

if [[ $username == "admin" && $password == "secret" ]]; then
	echo "valid user"
else
	echo "invalid user"
fi
\end{minted}


Pour plus d'informations sur les lignes de commandes bash, il existe un poly sur les commandes du terminal disponible sur le GitHub du club.
