\subsubsection{Histoire et Principe}

LaTeX a été créé en 1983, c'est un \textbf{langage et un système de composition de documents}.
Il sert principalement à de la mise en page simple de documents et a pour but de séparer le fond et la forme. 

LaTeX est devenu le langage privilégié pour les documents scientifiques du faite de sa simplicité. 

Pour rédigé du LaTeX, il faut uniquement se concentrer sur la structure logique du document et son contenu car le logiciel s'occupe de la mise en page automatiquement. 
Vous pouvez écrire en LaTeX grâce a diffèrent logiciel d’éditeur de texte comme TexMaker, Texlive, TeXworks, TexMacs, ... où il est possible de voir directement la mise en page en format PDF lorsqu'il es compiler.
Mais il est également possible de manipuler LaTeX simplement a partir d'un terminal ou de gedit.

L'évolution de LaTeX est assurée par la communauté d'utilisateurs qui regroupe des étudiants et des professeurs de mathématiques ou de physiques, comme des musiciens et ingénieurs en informatiques notamment.


\subsubsection{Différence avec Word}

Il faut d'abord savoir que les deux logiciels ne se comporte et ne s'utilise pas de la même façon. 

\begin{itemize}
\item La mise en page (images, figures, légendes, formules mathématiques, dessins, tableaux, ...) sur Word est une rude manipulation qui fais perdre du temps. LaTeX le fait tout seul, mais son interface austère fait "peur" aux debutant. Pourtant il suffit de lui dire quel type de documents on souhaite obtenir pour obtenir quelque chose de lisible et adapté avec les normes éditoriales ;
\item LateX est gratuit, contrairement à Word ;
\item Les formules mathématiques sont simple d'écriture sur LaTeX ;
\item Tout est modifiable et paramétrable avec LaTeX à n'importe quel moment (\textit{si au bout de la 100eme page vous vous rendez compte que vous voulez changer de police par exemple}) ;
\item La gestion de documents longs est intuitive sur LaTeX, contrairement a la complexité sur Word lorsqu'il faut gérer la mise en page identique ;
\item LaTeX peut générer automatiquement des bibliothèques ou table de matières beaucoup plus facilement que sur Word ;
\item Accèder à la création des PDF rapidement sur LaTeX ;
\end{itemize}


\subsubsection{Utilisation}

Le balisage est assez semblable au langage HTML. Il est donc possible de \textbf{créé ou modifier des macro-commandes} afin d'ajouter des raccourci. \textit{Par exemple pour regrouper plusieurs instructions en une.}
Comme ce langage a été créé avant la Unicode, tout caractères peut s'écrire en ASCII. 

Il y a tout d'abord tout un package a faire pour déterminer la langue d’écriture, les marges, polices d’écriture, couleur et taille de l’écriture style de document, ...
Tout ces \textbf{peuvent être trouver déjà fait sur Internet}, vous pouvez donc simplement les modifier.


Voici les principales instructions qui vous servirons : 

\begin{itemize}
\item Partie : $ \setminus part\lbrace \textit{nom\_ de\_ la\_ partie} \rbrace $ 

\item Section : $ \setminus section\lbrace \textit{nom\_ de\_ la\_ partie}\rbrace $ 

\item Sous-section : $ \setminus subsection\lbrace \textit{nom\_ de\_ la\_ partie} \rbrace $ 

\item Paragraphe : $ \setminus paragraph\lbrace \textit{nom\_ de\_ la\_ partie}\rbrace $

\item Saut de ligne : $ \setminus newline ou \setminus \setminus $ 

\item Liste : $ commencer par \setminus begin \lbrace itemize \rbrace puis pour chaque tiret \setminus item et conclure par \setminus end\lbrace itemize \rbrace $ \textit{ NB : pour modifier les puces il suffit de mettre la puces souhaitée entre $ [  ] $ apres $ \setminus item $} 
\newline pour des liste numérotées il faut écrire $\setminus begin\lbrace enumerate\rbrace $

\item très très petite écriture : $ \setminus scriptsize $

\item très petite écriture : $ \setminus footnotesize $

\item petite écriture : $ \setminus small $

\item grande écriture : $ \setminus large$ 

\item très grande écriture : $ \setminus LARGE $

\item très très grande écriture : $ \setminus huge $

\item Gras :  $ \setminus textbf\lbrace\rbrace $
\item Italique :  $ \setminus textit\lbrace\rbrace $
\item Penché :  $ \setminus textsl\lbrace\rbrace $
\item Machine à écrire :  $ \setminus texttt\lbrace\rbrace $
\item Exposant :  $ \setminus textsuperscript\lbrace\rbrace $
\item Encadré :  $ \setminus fbox\lbrace\rbrace $
\item Souligné :  $ \setminus ul\lbrace\rbrace $
\item Barré :  $ \setminus st\lbrace\rbrace $
\end{itemize}

différentes polices : bch, cmr, lmr, lmss, pag, pbk, phv, ...
