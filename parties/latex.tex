\subsubsection*{Histoire et Principe}

\paragraph{} \LaTeX{} a été créé en 1983, c'est un \textbf{langage et un
système de composition de documents}. Il sert principalement à de la mise en
page de documents et a pour but de séparer le fond et la forme.

\LaTeX{} est devenu le langage privilégié pour les documents scientifiques et
de beaucoup de livres.

\paragraph{} Pour rédiger du \LaTeX, il faut uniquement se concentrer sur la
structure logique du document et son contenu car le logiciel s'occupe de la
mise en page automatiquement. Vous pouvez écrire en \LaTeX{} grâce à différents
logiciels d’éditeurs de texte dédiés comme TexMaker, Texlive, TeXworks,
TexMacs\ldots Dans ces éditeurs, il est possible de voir directement la mise en
page en format PDF lorsqu'il est compilé. Mais il est également possible de
manipuler \LaTeX{} simplement à partir d'un terminal ou d'un éditeur de texte.

L'évolution de \LaTeX{} est assurée par la communauté d'utilisateurs qui
regroupe des étudiants et des professeurs de mathématiques ou de physique,
comme des musiciens et ingénieurs en informatique notamment.

\subsubsection*{Différences avec Word}

\paragraph{} Il faut d'abord savoir que les deux logiciels ne se comportent et
ne s'utilisent pas de la même façon.

\begin{itemize}
	\item La mise en page (images, figures, légendes, formules mathématiques,
		dessins, tableaux, ...) sur Word est une rude manipulation qui fait
		perdre du temps. \LaTeX{} le fait tout seul, mais son interface austère
		fait "peur" aux débutants. Pourtant il suffit de lui dire quel type de
		documents on souhaite obtenir pour obtenir quelque chose de lisible et
		adapté avec les normes éditoriales
	\item \LaTeX{} est gratuit (et même libre), contrairement à Word
	\item Les formules mathématiques sont simples d'écriture sur \LaTeX
	\item Tout est modifiable et paramétrable avec \LaTeX{} à n'importe quel
		moment (par exemple si au bout de la 100\ieme{} page vous vous rendez
		compte que vous voulez changer de police, uniquement sur tous les
		titres)
	\item La gestion de documents longs est intuitive sur \LaTeX ,
		contrairement à la complexité sur Word lorsqu'il faut gérer la mise en
		page identique
	\item \LaTeX{} peut générer automatiquement des bibliographies, tables de
		matières ou glossaires beaucoup plus facilement que sur Word
	\item Accéder à la création des PDF rapidement sur \LaTeX
\end{itemize}

\subsubsection*{Utilisation}

\paragraph{} Il est possible de \textbf{créer ou modifier des macro-commandes}
afin d'ajouter des raccourcis, par exemple pour regrouper plusieurs
instructions en une seule. Comme ce langage a été créé avant le Unicode, tous
les caractères peuvent s'écrire en ASCII (mais ce n'est plus nécessaire).

Il faut tout d'abord inclure les ``packages'' à utiliser pour déterminer la
langue d’écriture, les marges, polices d’écriture, couleur et taille de
l’écriture, style du document, etc. Tous ces packages \textbf{peuvent être
trouvés documentés sur Internet}, vous pouvez donc simplement les importer.

Voici les principales instructions qui vous serviront:

\begin{description}
	\item[Partie:] \mintinline{latex}{\part{nom de la partie}}
	\item[Section:] \mintinline{latex}{\section{nom de la section}}
	\item[Sous-section:] \mintinline{latex}{\subsection{nom de la sous-section}}
	\item[Paragraphe:] \mintinline{latex}{\paragraph{Nom optionel} contenu du paragraphe}
	\item[Paragraphe:] \mintinline{latex}{\newline ou \\}
	\item[Liste:] ~\\\begin{minted}{latex}
% Liste à point
\begin{itemize}
	\item Contenu
	\item de la
	\item liste
\end{itemize}

% Liste numérotée
\begin{enumerate}
	\item Contenu
	\item de la liste
	\item
		\begin{enumerate}
			\item Sous
			\item liste
		\end{enumerate}
\end{enumerate}

% Liste de descriptions
\begin{description}
	\item[Mot:] description
	\item[Autre mot:] autre description TEST
\end{description}
		\end{minted}
	\item[Mise en avant:] \mintinline{latex}{\emph{mon texte}}
	\item[Gras:] \mintinline{latex}{\textbf{mon texte}}
	\item[Italique:] \mintinline{latex}{\textit{mon texte}}
	\item[Machine à écrire:] \mintinline{latex}{\texttt{mon texte}}
	\item[Exposant:] \mintinline{latex}{\textsuperscript{mon texte}}
\end{description}

\paragraph{} Pour une excellente ressource sur le \LaTeX{} et un format plus
``tutoriel'':

\url{https://en.wikibooks.org/wiki/LaTeX}.
