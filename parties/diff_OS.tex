En informatique, un systeme d'exploitation est un ensemble de programmes qui dirige l'utilisation des capacites d'un ordinateur par des logiciels applicatifs.\newline


\subsection{Differents OS}
\begin{itemize}
\item MacOS (series d'interfaces graphiques basé sur l'operation des systemes developé par Apple pour leur Macintosh),
\item iOS (systeme d'operation pour mobile developper et distribuer par Apple pour iPhone et iPod),
\item \textbf{Linux} (c'est un Unix. systeme d'operation pour ordinateur assembler sous le modele du "free and open source software"),
\item Android (c'est un derivé de Linux. systeme d'operation designer pricipalement pour les mobiles tactiles comme les smartphones et les tablettes, initialement developper par Android),
\item Microsoft Windows (serie d'interface graphique developper et commercialiser par Microsft),
\item BSD/OS (reputaté pour reabiliter le role des serveurs, oganiser par les programmeur d'Unix, utiliser pour une utilisateur personnel du web).\newline
\end{itemize}


\subsection{differentes familles d'OS Linux} 

\begin{itemize}
\item \textbf{Debian/Ubuntu} : \textit{1993}
    \begin{itemize}
    \item developpé par SPI (\textit{Software in the Public Interest}),
	\item caractere non commercial et mode de gouvernance cooperatif,
	\item deja installer avec son noyau, ses pilotes, son programme d'installation de distribution, ses logiciels "utiles" (pour le WiFi, une navigateur web, ...) ,
	\item reuni une dizaine de sous-famille Debian avec le même noyau mais avec une architecture differente, dont les plus connus sont : KaliLinux, Kubunto, Raspbian, Ubuntu, Xubunto.\newline 
    \end{itemize}
   
\item \textbf{Red Hat/Cent OS/Fedora} : \textit{1994}
    \begin{itemize}
	\item developper et distribuer par l'entreprise Red Hat (entreprise dedier aux logiciels Open sources + distributeur de systeme d'exploitation Linux),
	\item plusieurs grosse distribution son issus de ce dev : Fedora, Enigma, ...
	\item principalement destinee aux serveurs des entreprises,
	\item voulait faire passer doucement les utilisateurs Windows sous Linux.\newline
	\end{itemize}

\item \textbf{Arch} : \textit{2002}
    \begin{itemize}
	\item accent sur la simplicité et legerete : parfait pour les utilisateurs avancés,
	\item pas d'outils graphiques : pour resoudre tout les problemes du systemes,
	\item contribution ouverte (OpenSource) tant que cela respecte sa philosophie.\newline
	\end{itemize}
	
\item \textbf{Suse/OpenSuse} : \textit{1994}
    \begin{itemize}
	\item distribution communautaire et commerciale,
	\item destine a l'utilisation en entreprise mais toujours en Open-Source,
	\item cycle de developpement long mais cycle de vie long,
	\item disponible par la vente (licence et mise a jour).
	\end{itemize}
\end{itemize}
