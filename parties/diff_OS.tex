\paragraph{} En informatique, un système d'exploitation est un ensemble de
programmes qui dirige l'utilisation des capacités d'un ordinateur par des
logiciels applicatifs.

\subsubsection{Différents types d'OS basés sur différents noyaux}

\begin{description}
	\item[MacOS:] série d'interfaces graphiques basée sur l'opération des
		systèmes développé par Apple pour leur Macintosh, originellement basé
		sur un système UNIX
	\item[iOS:] système d'exploitation pour mobile développé et distribué par
		Apple pour iPhone et iPod
	\item[GNU/Linux:] système d'exploitation pour ordinateur
		assemblé sous le modèle du "Free and Open Source Software" ou
		\textit{FOSS}
	\item[Android:] c'est un dérivé de GNU/Linux ; système d'opération désigné
		principalement pour les mobiles tactiles comme les smartphones
		et les tablettes, initialement développé par Android
	\item[Microsoft Windows:] série d'interfaces graphiques développée et
		commercialisée par Microsoft, basé originellement sur MS-DOS qui était
		développé par IBM
	\item[BSD:] réputé pour réhabiliter le rôle des serveurs, organisé par
		les programmeurs d'UNIX
\end{description}


\subsubsection{Différentes familles d'OS Linux}

\begin{description}
	\item[Debian/Ubuntu:] \textit{1993}
		\begin{itemize}
			\item développé par SPI (\textit{Software in the Public Interest})
			\item caractère non commercial et mode de gouvernance coopératif
			\item déjà installé avec son noyau, ses pilotes, son programme
				d'installation de distribution, ses logiciels "utiles" (pour le
				WiFi, un navigateur web, etc.)
			\item réunit une dizaine de sous-familles Debian avec
				principalement les mêmes composants mais avec des différences
				plus ou moins mineures. Les plus connues sont: KaliLinux,
				Kubuntu, Raspbian, Ubuntu, Xubuntu, etc.
		\end{itemize}

	\item[Red Hat/Cent OS/Fedora:] \textit{1994}
		\begin{itemize}
			\item développé et distribué par l'entreprise Red Hat (entreprise,
				rachetée par IBM en 2018, dédiée aux logiciels Open source +
				distributeur de système d'exploitation GNU/Linux)
			\item plusieurs distributions importantes en sont issues: Fedora,
				Enigma, etc.
			\item principalement destinées aux serveurs des entreprises
			\item Fedora voulait faire passer progressivement les utilisateurs
				Windows sous GNU/Linux
		\end{itemize}

	\item[Arch:] \textit{2002}
		\begin{itemize}
			\item accent sur la simplicité et légèreté et les utilisateurs
				avancés
			\item contribution ouverte (Open Source) tant que cela respecte sa
				philosophie
		\end{itemize}

	\item[Suse/OpenSuse:] \textit{1994}
		\begin{itemize}
			\item distribution communautaire et commerciale
			\item destinée à l'utilisation en entreprise mais toujours Open
				Source
			\item cycle de développement long mais cycle de vie long
			\item disponible à la vente (licence et mise à jour)
		\end{itemize}
\end{description}
