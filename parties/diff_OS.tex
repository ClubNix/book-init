En informatique, un système d'exploitation est un ensemble de programmes qui dirige l'utilisation des capacités d'un ordinateur
par des logiciels applicatifs.\newline

\subsubsection{Différents OS}

\begin{itemize}
\item MacOS (série d'interfaces graphiques basée sur l'opération des systèmes développé par Apple pour leur Macintosh),
\item iOS (système d'opérations pour mobile développé et distribué par Apple pour iPhone et iPod),
\item \textbf{Linux} (c'est un Unix ; système d'opérations pour ordinateur assemblé sous le modèle du "free and open source software"),
\item Android (c'est un dérivé de Linux ; système d'opération désigné principalement pour les mobiles tactiles comme les smartphones
et les tablettes, initialement développé par Android),
\item Microsoft Windows (série d'interfaces graphiques développée et commercialisée par Microsoft),
\item BSD/OS (réputé pour réhabiliter le rôle des serveurs, organisé par les programmeurs d'Unix, utilisé pour un utilisateur personnel du web).\newline
\end{itemize}


\subsubsection{différentes familles d'OS Linux}

\begin{itemize}
\item \textbf{Debian/Ubuntu} : \textit{1993}
    \begin{itemize}
    \item développé par SPI (\textit{Software in the Public Interest}),
	\item caractère non commercial et mode de gouvernance coopératif,
	\item déjà installé avec son noyau, ses pilotes, son programme d'installation de distribution, ses logiciels "utiles" (pour le WiFi, un navigateur web, ...) ,
	\item réunit une dizaine de sous-familles Debian avec le même noyau mais avec une architecture différente, dont les plus connues sont : KaliLinux, Kubunto, Raspbian, Ubuntu, Xubunto.\newline
    \end{itemize}

\item \textbf{Red Hat/Cent OS/Fedora} : \textit{1994}
    \begin{itemize}
	\item développé et distribué par l'entreprise Red Hat (entreprise dédiée aux logiciels Open source +
   distributeur de système d'exploitation Linux),
	\item plusieurs grosses distributions sont issues de ce dev : Fedora, Enigma, ...
	\item principalement destinée aux serveurs des entreprises,
	\item voulait faire passer doucement les utilisateurs Windows sous Linux.\newline
	\end{itemize}

\item \textbf{Arch} : \textit{2002}
    \begin{itemize}
	\item accent sur la simplicité et légèreté : parfait pour les utilisateurs avancés,
	\item pas d'outils graphiques : pour résoudre tous les problèmes du système,
	\item contribution ouverte (OpenSource) tant que cela respecte sa philosophie.\newline
	\end{itemize}

\item \textbf{Suse/OpenSuse} : \textit{1994}
    \begin{itemize}
	\item distribution communautaire et commerciale,
	\item destinée à l'utilisation en entreprise mais toujours en Open-Source,
	\item cycle de développement long mais cycle de vie long,
	\item disponible par la vente (licence et mise à jour).
	\end{itemize}
\end{itemize}
