\subsubsection*{Définition}

\paragraph{} Dans le domaine informatique, le terme ``log'' désigne un type de
fichier, ou une entité équivalente, dont la mission principale consiste à
stocker un historique des événements. Diminutif de \textit{logging}, le terme
peut être traduit en français par "journal". Le log s'apparente ainsi à un
journal de bord horodaté, qui ordonne les différents événements qui se sont
produits sur un ordinateur, un serveur, etc.

\paragraph{} Ils s'avèrent utiles pour comprendre la provenance d'une erreur en
cas de bug. Ils permettent également d'établir des statistiques, comme le
nombre de connexions à un site Web ou à un serveur, le nombre d'échec
d'authentification, etc.

\paragraph{} S'il s'agit d'un serveur Web, un fichier log va par exemple
enregistrer la date et l'heure de la tentative d'accès, l'adresse IP du client,
le fichier cible, le système d'exploitation utilisé, le navigateur, la réponse
du serveur à cette requête, éventuellement le type d'erreur rencontré...

\paragraph{} Les fichiers de log peuvent contenir des informations
confidentielles
