\subsubsection{Définition}
Dans le domaine informatique, le terme log désigne un type de fichier, ou une entité équivalente, dont la mission principale consiste à stocker un historique des événements. Diminutif de logging, le terme peut être traduit en français par "journal". Le log s'apparente ainsi à un journal de bord horodaté, qui ordonne les différents événements qui se sont produits sur un ordinateur, un serveur, etc. Il permet ainsi d'analyser heure par heure, voire minute par minute, l'activité interne d'un processus.


Chaque action d'un système informatique (ouverture d'une session, installation d'un programme, navigation sur Internet...) produit un fichier log. Diminutif de logging, le terme peut être traduit en français par "journal" .Ces fichiers textes listent chronologiquement les événements exécutés. Ils s'avèrent utiles pour comprendre la provenance d'une erreur en cas de bug.
Ils permettent également d'établir des statistiques de connexions à un site Web ou à un serveur,
S'agissant d'un serveur Web, un fichier log va enregistrer la date et l'heure de la tentative d'accès, l'adresse IP du client, le fichier cible, le système d'exploitation utilisé, le navigateur, la réponse du serveur à cette requête, éventuellement le type d'erreur rencontré...

Les fichiers logs peuvent contenir des informations confidentielles

\begin{itemize}
	\item L’heure et date de consultation
	\item Le nombre de consultations
	\item La durée de la session
	\item L’adresse IP et le nom d‘hôte de l’utilisateur
	\item Les informations sur le client demandeur (en général le navigateur)
	\item Le moteur de recherche utilisé, dont les requêtes
	\item Le système d’exploitation utilisé
	\item Une entrée classique d’un fichier log d’un serveur Web se présente comme ci-dessous: \textit{183.121.143.32 - - $ [ $18/Mar/2003:08:04:22 +0200 $ ] $ "GET /images/logo.jpg HTTP/1.1" 200 512 "http://www.wikipedia.org/" "Mozilla/5.0 (X11; U; Linux i686; de-DE;rv:1.7.5)"}
\end{itemize}
