\subsection{UNIX}

\textbf{Unix} est un \textbf{systeme d'exploitation} multi-taches et multi-utilisateurs créé en \textbf{1969}. Il repose sur un interpretateur (le shell) et sur plusieurs petits utilisateurs (avec des actions specifiques defini lors du code). C'est une \textbf{marque deposée de l'OpenGroup}

Son nom est un derivé de "Unics" (\textit{Uniplexed Information and Computing Service}). C'est un jeu de mot avec "Multics" (un autre noyau) qui vise a offrir simultanement plusieurs services a un ensemble d'utilisateur.

Unix a donner naissance a une \textbf{famille de systemes} comme BSD, GNU/LINUX, iOS ou encore MacOS, eux-même divisé en variante de système d'exploitation aux \textbf{normes POSIX}. Cette norme technique standardise les interfaces de programmation des logiciels. 
Il faut savoir que la casi totalité des systèmes pc ou mobiles (à l'exception des Windows) sont basés sur le noyau de UNIX (y compris Apple). 

Comme dit precedement, Unix est un noyau, \textbf{il doit donc etre utilisé avec d'autres logiciels} car il lui faut un systeme d'exploitation. \newline



\subsection{GNU : "GNU's Not UNIX"}

GNU est un \textbf{projet des annees 1990} lancé par Richard STALLMAN. C'est un \textbf{système d'exploitation libre} compatible avec le noyau UNIX. Il utilise des logiciels libres issus d'autres projets tels que le noyaux Linux (\textit{voir GNU/LINUX}). 

Sa \textbf{philosophie} est de maintenir intacte les \textbf{tradictions hackers de partage}] dans un monde de plus en plus marqué par l'empreinte du droit d'auteur. Il se bat donc pour une \textbf{libre diffusion des connaissances}. "GNU vise à ne laisser l'homme devenir ni l'esclave de la machine ni de ceux qui auraient l'exclusivité de sa programmation". 

Il est \textbf{utilisable et partagable librement par tous}, ainsi chacun complete petit a petit l'architecture initale de GNU pour le performer. C'est dans ce contexte que le projet invite la communauté de hacker a le rejoindre et a participer a son developement. 

Il faut savoir qu'il est composé : 
\begin{itemize}
	\item d'un editeur de texte (emacs),
	\item d'un compilateur tres performant (gcc),
	\item d'un debogueur (gdb), 
	\item d'un langage de script (bash),
	\item des bibliotheques de systemes (glibc),
	\item (plus tard noyau ramener par un projet linux ). 
\end{itemize}
Malgres tout ces composants, GNU est tjrs \textbf{incomplet} : son noyau est immature et imcompatible avec certains de ses composants. \newline



\subsection{GNU/LINUX}

GNU/LINUX a été créé en \textbf{1991}. Il est \textbf{initié par le projet Debian} et la naissance du noyau Linux. Il credite donc a la fois Linux et GNU \textit{mais l'usage de Linux est plus connus au grand publique}. Il est alors toujours baser sur le \textbf{mouvement du logiciel libre et du mode operatoire du hacker}. 

Cet associement a eu lieu pour combler le vide causé par le développement inachevé de GNU (\textit{voir GNU}). Il est utilisé sur la plupart des telepones portables comme sur les super-ordinateurs (par exemple, android équipe 80\% de ses smartphones).Ce projet fut un grand d'impacte dans le monde des serveurs informatiques.

GNU/LINUX veut casser le faite qu'a l'origine il fallait des connaissances solide en informatiques pour utiliser un système d'exploitation (\textit{pas d'interface graphique et besoin d'installer toutes les applications soit même}).Il a donc été un important vecteur de \textbf{popularisation du mouvement de l'open source}.

Il a eu des \textbf{centaines de miliers de redistributions} avec des versions differentes pour plaire a tout les gouts (en fonction des besoins, configration, securite, ...\textit{voir differents OS}). GNU/LINUX a remplacé d'autres systemes de type Unix et/ou evite l'achat de licence Windows (qui est trés cher à l'achat). Aujourd'hui on peut retrouver tout les equivalent des logiciel/applications qu'il y a sous Windows mais en Open Sources.\newline



\subsection{Logiciel libre}

Tout logiciel libre est defini par la \textbf{licence publique generale GNU} (\textit{appelé GPL}) qui rend les logiciels independants de tout editeurs en les encourage a l'entraide et le partage.

La licence generale est defini par 4 lois principales :
\begin{itemize}
	\item utiliser un logiciel sans restrictions,
	\item etudier le logiciel,
	\item modifier pour l'adapter aux besoins des utilisateurs,
	\item reditribuer sous certaines conditions precises.
\end{itemize}
Cela a bcp d'avantages comme \textbf{correction rapide de bogues et de failles de securites}.

\textbf{Attention,} un logiciel libre n'est pas necessairement gratuit et inversement un logiciel gratuit n'est pas forcement libre ! \newpage


 

 
En informatique, un systeme d'exploitation est un ensemble de programmes qui dirige l'utilisation des capacites d'un ordinateur par des logiciels applicatifs.\newline


\subsection{Differents OS}
\begin{itemize}
\item MacOS (series d'interfaces graphiques basé sur l'operation des systemes developé par Apple pour leur Macintosh),
\item iOS (systeme d'operation pour mobile developper et distribuer par Apple pour iPhone et iPod),
\item \textbf{Linux} (c'est un Unix. systeme d'operation pour ordinateur assembler sous le modele du "free and open source software"),
\item Android (c'est un derivé de Linux. systeme d'operation designer pricipalement pour les mobiles tactiles comme les smartphones et les tablettes, initialement developper par Android),
\item Microsoft Windows (serie d'interface graphique developper et commercialiser par Microsft),
\item BSD/OS (reputaté pour reabiliter le role des serveurs, oganiser par les programmeur d'Unix, utiliser pour une utilisateur personnel du web).\newline
\end{itemize}


\subsection{differentes familles d'OS Linux} 

\begin{itemize}
\item \textbf{Debian/Ubuntu} : \textit{1993}
    \begin{itemize}
    \item developpé par SPI (\textit{Software in the Public Interest}),
	\item caractere non commercial et mode de gouvernance cooperatif,
	\item deja installer avec son noyau, ses pilotes, son programme d'installation de distribution, ses logiciels "utiles" (pour le WiFi, une navigateur web, ...) ,
	\item reuni une dizaine de sous-famille Debian avec le même noyau mais avec une architecture differente, dont les plus connus sont : KaliLinux, Kubunto, Raspbian, Ubuntu, Xubunto.\newline 
    \end{itemize}
   
\item \textbf{Red Hat/Cent OS/Fedora} : \textit{1994}
    \begin{itemize}
	\item developper et distribuer par l'entreprise Red Hat (entreprise dedier aux logiciels Open sources + distributeur de systeme d'exploitation Linux),
	\item plusieurs grosse distribution son issus de ce dev : Fedora, Enigma, ...
	\item principalement destinee aux serveurs des entreprises,
	\item voulait faire passer doucement les utilisateurs Windows sous Linux.\newline
	\end{itemize}

\item \textbf{Arch} : \textit{2002}
    \begin{itemize}
	\item accent sur la simplicité et legerete : parfait pour les utilisateurs avancés,
	\item pas d'outils graphiques : pour resoudre tout les problemes du systemes,
	\item contribution ouverte (OpenSource) tant que cela respecte sa philosophie.\newline
	\end{itemize}
	
\item \textbf{Suse/OpenSuse} : \textit{1994}
    \begin{itemize}
	\item distribution communautaire et commerciale,
	\item destine a l'utilisation en entreprise mais toujours en Open-Source,
	\item cycle de developpement long mais cycle de vie long,
	\item disponible par la vente (licence et mise a jour).
	\end{itemize}
\end{itemize}