\subsubsection*{Définition}

\paragraph{} Git est le logiciel de gestion de version le plus couramment
utilisé dans le monde (il en existe d'autres comme SVN mais beaucoup moins
utilisés), il fût développé en 2005 par Linus Torvalds (créateur du noyau
Linux).

\paragraph{} Git est un outil de versionnement, c'est à dire que lorsqu'une ou
plusieurs personnes sont sur un même projet, Git va garder les différentes
versions dans le temps.

Un projet est constitué de plusieurs fichiers amenés à évoluer dans le temps.
Git va conserver les informations sur Qui a modifié Quoi, Quand et Pourquoi.
Cet outil est donc primordial durant la conception d'un projet à plusieurs et
même seul (lorsque l'on revient sur du code écrit il y a 3 mois on ne sait pas
toujours à quoi il correspond).

\paragraph{} Attention, il ne faut pas confondre Git avec GitHub et GitLab.
GitHub et GitLab vont se servir de Git mais ils vont aussi rajouter une
interface graphique et vont permettre le stockage sur un serveur distant. De
plus il ne faut pas confondre GitHub et Gitlab, GitHub est géré par une
entreprise privée qui entre temps a été rachetée par Microsoft (entreprise
vouée à l'échec) et l'autre est une entreprise qui fourni aussi son logiciel,
qui est partiellement Open Source.

\subsubsection*{Commandes de base}

Voir le poly Git du club qui est sur GitHub.
