\subsubsection{Définition}

Git est le logiciel de gestion de version décentralisé (pas forcément besoin
d'être en ligne) le plus couramment utilisé dans le monde (il en existe d'autres
comme SVN mais beaucoup moins utilisés), il fût développé en 2005 par Linus
Torvalds (créateur du noyau linux).\newline

Git est un outil de versionnement, c'est à dire que lorsque plusieurs personnes
sont sur un même projet, Git va garder les différentes versions dans le temps.

En réalité, Git va conserver les modifications effectuées, ce qui empêche de
prendre trop de place.



Un projet constitue plusieurs fichiers amenés à évoluer dans le temps. Git va
conserver les informations sur Qui a modifié Quoi, Quand et Pourquoi.
Cet outil est donc primordial durant la conception d'un projet à plusieurs et
même seul (lorsque l'on revient sur du code écrit il y a 3 mois on ne sait pas
toujours à quoi il correspond).\newline


Attention, il ne faut pas confondre Git avec GitHub et GitLab. GitHub et GitLab
vont se servir de Git mais ils vont aussi rajouter une interface graphique et
vont permettre le stockage sur un serveur distant. De plus il ne faut pas
confondre Github et Gitlab, Github est géré par une entreprise privée qui entre
temps a été rachetée par Microsoft (entreprise vouée à l'échec) et l'autre est
une entreprise qui fourni aussi son logiciel en Open Source.

\subsubsection{Commandes de base}
Voir le poly Git du club qui est sur le Github.
