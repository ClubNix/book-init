\subsubsection{Définition}
Git est un logiciel de gestion de version décentralisé (pas forcément besoin d'être en ligne) le plus couramment utilisé dans le monde
(il en existe d'autres comme SVN mais beaucoup moins utilisés), il fut développé en 2005 par Linus Torvalds (créateur du noyau linux).
Git est un outil de versionning, c'est à dire que lorsque plusieurs personnes sont sur un même projet, par exemple ils rédigent un code à plusieurs.
 Un projet constitue plusieurs fichiers amenés à évoluer dans le temps. Git va les prévenir de Qui a modifié le fichier, Quand, Quoi et Pourquoi.
 Cet outil est donc primordial durant la conception d'un projet à plusieurs et même seul
 (lorsque l'on revient sur du code écrit il y a 3 mois on ne sait pas toujours à quoi il correspond).
 Lorsqu'on travaille à plusieurs sur un même projet Git fait une copie et une fois modifié, il rassemble tous les fichiers en un.

Attention, il ne faut pas confondre Git avec Github et Gitlab. Github et Gitlab vont rajouter une interface graphique et vont permettre
 de le partager. De plus il ne faut pas confondre Github et Gitlab, Github est géré par une entreprise privée qui entre temps a été
 rachetée par Microsoft (entreprise vouée à l'échec) et l'autre est opensource.

\subsubsection{Commandes de base}
\begin{itemize}
	\item Récupérer un projet \textit{git clone (adresse repository)}
	\item Prendre en compte un changement (créer un point de sauvegarde) \textit{git commit -m(message)}
	\item Récupérer quelque chose qui aurait changé en ligne \textit{git pull}
	\item Mettre en ligne \textit{git push}
\end{itemize}
