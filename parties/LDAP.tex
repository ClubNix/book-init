\usepackage[french]{babel}
\subsection{Le LDAP, ou \textit{Lightweigth Directory Acess Protocol}}
Lorsqu'un utilisateur tente de se connecter à un ordinateur, le LDAP recois une combinaison login/mot de passe et donne ensuite ou non l'autorisation de se connecter.
\subsection{Introduction}
Le LDAP est un protocole crée en 1995, succedant au protocol DAP et permettant l'accès et la modification de base de donnée sur les utilisateurs d'un reseau. Un LDAP sert notement à se connecter ou se déconnecter d'un serveur hébergant le LDAP afin d'y être identifié, mais aussi à chercher des informations, les comparer, ajouter des utilisateurs...
\subsection{Le LDAP du club Nix ou de l'école}
Le LDAP du club NIX ou de l'école permet de se connecter et de s'identifier afin de récuperer ses informations, quelque soit la machine utilisée.
\subsection{Les entrées}
Un LDAP définit l'accès aux entrées ( ou utilisateurs la plus part du temps).
Le LDAP ne peut gérer que des entrées. Une entrée peut etre un nom d'utilisateur, un périférique ou encore des parametres. Il existe 2 types d'entrées, les entrées normales et les entrées opérationnelles :
\begin{itemize}
    \item Les entrées classiques, telles le nom d'utilisateur ou la date d'aniveraires sont des entrées dites classiques.
    \item Les entrées opérationnelles, tels les parametres, les dates de modification, qui ne sont accessibles et utilisables uniquement par le serveur.
\end{itemize}
Une entrée est définie par son nom, ou DN pour \textit{Distinguished Name}, composé d'une serie de clé et de valeur de ces clé. Par exemple, la clé \textit{uid} définit le nom d'utilisateur et la clé \textit{cn} définit le nom. Pour ces deux clé, une entrée serait sous la forme : \textit{uid=utilisateur,cn=nix}
\subsection{L'arborescence}
Un serveur hébergant un LDAP est organisé selon une arborescence, comme un systeme de ficher, dans lequel chaque branche correspond à une entrée. Une branche situé à la racine sera appelé racine ou root en anglais. Le scheme d'une clé correspond a l'ensemble des valeurs des attributs ou valeurs attribué aux clés.
Les annuaires LDAP répondent à certaines règles de structure :
\begin{itemize}
    \item Un annuaire est un arbre d'entrées;
    \item Une entrée est constituée d'un ensemble d'attributs;
    \item Un attribut possède un nom, un type et une ou plusieurs valeurs;
    \item Les attributs sont définis dans des schémas.
\end{itemize}
\subsection{Chercher des informations d'un LDAP}
Le protocole LDAP fournit un ensemble de fonction permettant d'interroger le serveur sur lequel est herbergé le serveur LDAP afin de modifer, ajouter ou supprimer des entrées. On peut citer notement *add* pour ajouter une entrée, *delete* pour la supprimer, ou *rename* pour la renomer, afin de modifier l'arborescence du LDAP.
\subsection{Les principales commandes du LDAP}
Voici une lise des principales clés d'un LDAP :
\begin{itemize}
    \item \textit{userid} (userid), il s'agit d'un identifiant unique obligatoire;
    \item \textit{cn} (common name), il s'agit du nom de la personne;
    \item \textit{givenname}, il s'agit du prénom de la personne;
    \item \textit{sn} (surname), il s'agit du surnom de la personne;
    \item \textit{o} (organization), il s'agit de l'entreprise de la personne;
    \item \textit{u} (organizational unit), il s'agit du service de l'entreprise dans laquelle la personne travaille.
    \item \textit{mail}, il s'agit de l'adresse de courrier électronique de la personne
\end{itemize}

Voici une liste des principales commandes d'un LDAP :
\begin{itemize}
    \item \textit{abandon} : Abandonne l'opération précédemment envoyées au serveur;
    \item \textit{add} : Ajoute une entrée au répertoire;
    \item \textit{bind} : Initie une nouvelle session sur le serveur LDAP;
    \item \textit{compare} : Compare les entrées d'un répertoire selon des critères;
    \item \textit{delete} : Supprime une entrée d'un répertoire;
    \item \textit{extended} : Effectue des opérations étendues;
    \item \textit{rename} : Modifie le nom d'une entrée;
    \item \textit{search} : Recherche des entrées d'un répertoire;
    \item \textit{unbind} : Termine une session sur le serveur LDAP.
\end{itemize}