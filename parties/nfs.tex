\subsubsection*{Introduction}

\paragraph{} Un NFS, ou \textit{Network File System}, à traduire par
\textit{Système de fichiers en réseau}, est un protocole qui permet à un client
à accéder à des fichiers/dossiers stockés sur un ou plusieurs serveurs. Ainsi
le(s) serveur(s) stocke les fichiers et les dossiers, tandis que les clients
donne l'impression d'avoir les fichiers et les dossiers présents sur la
machine.

\subsubsection*{Le NFS du Club*Nix ou de l'ESIEE}

\paragraph{} Le Club*Nix et l'ESIEE possèdent un NFS, permettant aux postes de
travail d'avoir accès aux fichiers des élèves/membres.

\paragraph{} Le NFS transmet aussi les informations sur les utilisateurs des
fichiers. Ainsi, un serveur NFS est souvent utilisé avec un serveur LDAP pour
gérer les utilisateurs (c.f. section \ref{subsec:ldap})

\subsubsection*{La sécurité du NFS}

Il existe actuellement 4 versions de NFS. Dans les 3 premières versions, le
protocole NFS n'était pas sécurisé et permettait l'accès aux fichiers de
n'importe qui ayant un accès réseau au serveur. Depuis la version 4.1, le
protocole NFS défini un système d'authentification et de chiffrement des
données transmises sur le réseau.
