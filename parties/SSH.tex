\subsection{Définition}

\paragraph{} SSH signifie "Secure Shell", c'est un protocole sécurisé pour les
communications à distance créé en 1995.

\paragraph{} Pour rappel, un Shell va permettre de dialoguer avec une machine
ou un serveur (grâce au terminal qui est une application graphique) via
l'exécution de différentes commandes qui retourneront des informations. Il
existe toutes sortes de Shell mais le plus utilisé reste Bash.

Un shell va donc nous permettre d'administrer nos serveurs Linux,
en local, c'est-à-dire lorsque l'on se trouve physiquement en face de notre
serveur, mais aussi à distance! Notamment grâce au Secure Shell (SSH).

\paragraph{} L'administration à distance est aujourd'hui vitale lorsque l'on
gère un seul serveur, comme des milliers qui sont la plupart du temps
difficiles d'accès car ils sont stockés dans des datacenter et isolés
géographiquement.

Autrefois, d'autres protocoles étaient utilisés pour accéder à distance à un
serveur Linux. Le protocole Telnet a pendant longtemps été utilisé, il permet
également d'accéder à distance à une machine Linux, mais Telnet est aujourd'hui
délaissé au profit de SSH et cela pour une raison très simple: son manque de
sécurité. A l'inverse, SSH va créer un tunnel sécurisé entre la machine locale
et la machine/serveur distant (il va crypter les données).

Comme dit précédemment nous utilisons SSH surtout pour faire le lien entre nos
machines et nous.

\subsection{Les Commandes De Bases}
\begin{itemize}
	\item Toutes les commandes du terminal sont applicables
	\item Connexion à la machine distante avec le login john:\\
		\mintinline{bash}{ssh john@remotehost.example.com}
	\item Création d'une paire de clef publique/privée:\\
		\mintinline{bash}{ssh-keygen -t rsa -b 4096 -C "your_email@example.com"}\\
		Cela va permettre de se connecter aux machines distantes possédant
		notre clef publique de manière plus sécurisée et sans mot de passe.
	\item Copie de la clé publique sur la machine distante:\\
		\mintinline{bash}{ssh-copy-id -i ~/.ssh/id_dsa.pub john@remotehost.example.com}
	\item la commande \textbf{scp} va permettre de copier des fichiers entre la machine locale et le
		serveur en utilisant le protocole SSH:\\
		\mintinline{bash}{scp my_file john@remotehost.example.com:/home/john/a_folder}\\
		pour un seul fichier et:\\
		\mintinline{bash}{scp -r my_folder john@remotehost.example.com:/home/john/another_folder} pour un dossier
		(ajout de l’argument de récursivité). Il est indispensable de préciser le
		dossier de destination (après les ':'). Avec cette commande il est
		possible de copier dans les deux sens: de notre machine au serveur, ou
		du serveur à notre machine, selon l'ordre des arguments de la commande.
\end{itemize}
