\subsubsection{Définition}

SSH signifie "Secure Shell" est un protocole sécurisé pour les communications à distance créé en 1995 par Tatu Ylönen.
Pour rappel, un Shell va permettre de dialoguer avec une machine ou un serveur (grâce au terminal qui est une application graphique
 du Shell) via l'exécution de différentes commandes qui retourneront des informations. Il existe toutes sortes de Shell
 mais le plus utilisé reste bash.
Un shell va donc nous permettre d'administrer nos serveurs Linux, en local, c'est-à-dire lorsque l'on se trouve physiquement
 en face de notre serveur, mais aussi à distance ! Notamment grâce à Secure Shell (SSH).
L'administration à distance est aujourd'hui vitale lorsque l'on gère un seul serveur, comme des milliers qui sont
la plupart du temps difficiles d'accès car ils sont stockés dans des datacenter et isolés géographiquement.
Autrefois, d'autres protocoles étaient utilisés pour accéder à distance à un serveur Linux. Le protocole Telnet a
pendant longtemps été utilisé, il permet également d'accéder à distance à une machine Linux, mais Telnet est aujourd'hui
délaissé au profit de SSH et cela pour une raison très simple : son manque de sécurité. SSH va créer un tunnel entre l'utilisateur
et la machine (il va crypter les données).

Comme dit précédemment nous utilisons SSH surtout pour faire le lien entre nos serveurs et nous.

\subsubsection{Les Commandes De Bases}
\begin{itemize}
	\item Toutes les commandes du terminal sont applicables
	\item Connexion à la machine distante avec le login john 'ssh john@remotehost.example.com' ou 'ssh -l john remotehost.example.com'
	\item Copie de la clé publique sur la machine distante 'ssh-copy-id -i \~/.ssh/id\_dsa.pub john@remotehost.example.com'
	\item la commande \href{scp} va permettre de copier des fichiers entre le serveur et le client ssh de manière sécurisée \href{scp my\_file john@remotehost.example.com:/home/john/a\_folder}
	pour un seul fichier et \href{scp -r my\_folder john@remotehost.example.com:/home/john/another\_folder} pour un dossier
	(ajout de l’argument de récursivité). Il est indispensable de préciser le dossier de destination (après les ':').
\end{itemize}
