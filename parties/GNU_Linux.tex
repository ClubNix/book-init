\subsubsection{UNIX}

\textbf{Unix} est un \textbf{systeme d'exploitation} multi-tâches et multi-
utilisateurs créé en \textbf{1969}.
Il repose sur un interprétateur (le shell) et sur plusieurs petits utilisateurs
(avec des actions spécifiques selon son groupe).

C'est une \textbf{marque deposée de l'OpenGroup}.
Son nom est un dérivé de "Unics" (\textit{Uniplexed Information and Computing
Service}). C'est un jeu de mot avec "Multics"
(un autre noyau) qui vise à offrir simultanément plusieurs services à un
ensemble d'utilisateurs.

Unix a donné naissance à une \textbf{famille de systèmes} comme BSD, GNU/LINUX,
iOS ou encore MacOS, eux-même divisés en variantes de systèmes d'exploitation
aux \textbf{normes POSIX}. Cette norme technique standardise les interfaces de
programmation des logiciels.\newline

Il faut savoir que la quasi-totalité des systèmes PC ou mobiles (à l'exception
des systèmes Windows) sont basés sur le noyau de UNIX (y compris ceux de la
marque Apple).
Comme dit précédemment, Unix est un noyau, \textbf{il doit donc être utilisé
avec d'autres logiciels} car il lui faut un système d'exploitation.

\subsubsection{GNU : "GNU's Not UNIX"}

GNU est un \textbf{projet des années 1990} lancé par Richard STALLMAN. C'est un
\textbf{système d'exploitation libre} compatible avec le noyau UNIX. Il utilise
des logiciels libres issus d'autres projets tels que le noyaux Linux (\textit{
voir GNU/LINUX}).
Sa \textbf{philosophie} est de maintenir intacts les \textbf{tradictions hackers
de partage} dans un monde de plus en plus marqué par l'empreinte du droit
d'auteur. Il se bat donc pour une \textbf{libre diffusion des connaissances}.
"GNU vise à ne laisser l'homme devenir ni l'esclave de la machine ni de ceux qui auraient l'exclusivité de sa programmation".\newline

Il est \textbf{utilisable et partageable librement par tous}, ainsi chacun
complète petit a petit l'architecture initiale de GNU pour le rendre meilleur.
C'est dans ce contexte que le projet invite la communauté de hackers à le
rejoindre et à participer à son développement.
Il faut savoir qu'il est composé :
\begin{itemize}
	\item d'un éditeur de texte (emacs),
	\item d'un compilateur très performant (gcc),
	\item d'un débogueur (gdb),
	\item d'un langage de script (bash),
	\item de bibliothèques de systèmes (glibc),
	\item (plus tard d'un noyau ramené par le projet Linux).
\end{itemize}
Malgré tous ces composants, GNU est toujours \textbf{incomplet} : son noyau est immature et incompatible avec certains de ses composants. \newline


\subsubsection{GNU/LINUX}

GNU/LINUX a été créé en \textbf{1991}. Il est \textbf{initié par le projet
Debian} et la naissance du noyau Linux.
Il crédite donc à la fois Linux et GNU \textit{mais l'usage de Linux est plus
connu au grand public}.

Il est alors toujours basé sur le \textbf{mouvement du logiciel libre et du mode
opératoire du hacker}.
Cette association a eu lieu pour combler le vide causé par le développement
inachevé de GNU (\textit{voir GNU}).\newline

Il est utilisé sur la plupart des téléphones portables comme sur les
super-ordinateurs. Ce projet eut un grand impact dans le monde des serveurs
informatiques.
GNU/LINUX veut casser le fait qu'à l'origine il fallait des connaissances
solides en informatique pour utiliser un système d'exploitation (\textit{pas
d'interface graphique et besoin d'installer toutes les applications soi-même}).

Il a donc été un important vecteur de \textbf{popularisation du mouvement de
l'open source}.
Il a eu des \textbf{centaines de milliers de redistributions} avec des versions
différentes pour plaire à tous les goûts (en fonction des besoins,
configurations, sécurité, ...\textit{voir différents OS}).
GNU/LINUX a remplacé d'autres systèmes de type Unix et/ou évite l'achat d'une
licence Windows (qui est très chère à l'achat).\newline

Aujourd'hui on peut retrouver tous les équivalents des logiciels/applications
qu'il y a sous Windows mais en Open Source.\newline
