\subsection{UNIX}

\textbf{Unix} est un \textbf{systeme d'exploitation} multi-taches et multi-utilisateurs créé en \textbf{1969}. Il repose sur un interpretateur (le shell) et sur plusieurs petits utilisateurs (avec des actions specifiques defini lors du code). C'est une \textbf{marque deposée de l'OpenGroup}
Son nom est un derivé de "Unics" (\textit{Uniplexed Information and Computing Service}). C'est un jeu de mot avec "Multics" (un autre noyau) qui vise a offrir simultanement plusieurs services a un ensemble d'utilisateur.
Unix a donner naissance a une \textbf{famille de systemes} comme BSD, GNU/LINUX, iOS ou encore MacOS, eux-même divisé en variante de système d'exploitation aux \textbf{normes POSIX}. Cette norme technique standardise les interfaces de programmation des logiciels. 
Il faut savoir que la casi totalité des systèmes pc ou mobiles (à l'exception des Windows) sont basés sur le noyau de UNIX (y compris Apple). 
Comme dit precedement, Unix est un noyau, \textbf{il doit donc etre utilisé avec d'autres logiciels} car il lui faut un systeme d'exploitation. \newline


\subsection{GNU : "GNU's Not UNIX"}

GNU est un \textbf{projet des annees 1990} lancé par Richard STALLMAN. C'est un \textbf{système d'exploitation libre} compatible avec le noyau UNIX. Il utilise des logiciels libres issus d'autres projets tels que le noyaux Linux (\textit{voir GNU/LINUX}). 
Sa \textbf{philosophie} est de maintenir intacte les \textbf{tradictions hackers de partage}] dans un monde de plus en plus marqué par l'empreinte du droit d'auteur. Il se bat donc pour une \textbf{libre diffusion des connaissances}. "GNU vise à ne laisser l'homme devenir ni l'esclave de la machine ni de ceux qui auraient l'exclusivité de sa programmation". 
Il est \textbf{utilisable et partagable librement par tous}, ainsi chacun complete petit a petit l'architecture initale de GNU pour le performer. C'est dans ce contexte que le projet invite la communauté de hacker a le rejoindre et a participer a son developement. 
Il faut savoir qu'il est composé : 
\begin{itemize}
	\item d'un editeur de texte (emacs),
	\item d'un compilateur tres performant (gcc),
	\item d'un debogueur (gdb), 
	\item d'un langage de script (bash),
	\item des bibliotheques de systemes (glibc),  
	\item (plus tard noyau ramener par un projet linux ). 
\end{itemize}
Malgres tout ces composants, GNU est tjrs \textbf{incomplet} : son noyau est immature et imcompatible avec certains de ses composants. \newline


\subsection{GNU/LINUX}

GNU/LINUX a été créé en \textbf{1991}. Il est \textbf{initié par le projet Debian} et la naissance du noyau Linux. Il credite donc a la fois Linux et GNU \textit{mais l'usage de Linux est plus connus au grand publique}. Il est alors toujours baser sur le \textbf{mouvement du logiciel libre et du mode operatoire du hacker}. 
Cet associement a eu lieu pour combler le vide causé par le développement inachevé de GNU (\textit{voir GNU}). Il est utilisé sur la plupart des telepones portables comme sur les super-ordinateurs (par exemple, android équipe 80\% de ses smartphones).Ce projet fut un grand d'impacte dans le monde des serveurs informatiques.
GNU/LINUX veut casser le faite qu'a l'origine il fallait des connaissances solide en informatiques pour utiliser un système d'exploitation (\textit{pas d'interface graphique et besoin d'installer toutes les applications soit même}).Il a donc été un important vecteur de \textbf{popularisation du mouvement de l'open source}.
Il a eu des \textbf{centaines de miliers de redistributions} avec des versions differentes pour plaire a tout les gouts (en fonction des besoins, configration, securite, ...\textit{voir differents OS}). GNU/LINUX a remplacé d'autres systemes de type Unix et/ou evite l'achat de licence Windows (qui est trés cher à l'achat). Aujourd'hui on peut retrouver tout les equivalent des logiciel/applications qu'il y a sous Windows mais en Open Sources.\newline
