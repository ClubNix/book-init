\subsubsection*{UNIX}

\paragraph{} \textbf{UNIX} est un \textbf{systeme d'exploitation} multi-tâches
et multi-utilisateurs créé en \textbf{1969}. Il est composé de plusieurs
composants:

\begin{description}
	\item[Le noyau:] aussi appelé ``kernel'' en anglais. C'est un logiciel
		central au système d'exploitation; il fait le lien entre le matériel et
		les logiciels habituels.
	\item[Un environnement de développement:] ou tout simplement des outils qui
		permettent de créer de nouveaux logiciels/outils.
	\item[Des commandes] qui permettent d'exécuter des actions qu'on attend
		d'un système d'exploitation, comme manipuler les fichiers, aller sur
		internet, etc.
	\item[De la documentation] qui explique le fonctionnement du système et des
		outils fournis.
\end{description}

\paragraph{} C'est une \textbf{marque deposée de l'OpenGroup}. Son nom est un
dérivé de "Unics" (\textit{Uniplexed Information and Computing Service}). C'est
un jeu de mot avec "Multics" (un autre système) qui vise à offrir simultanément
plusieurs services à un ensemble d'utilisateurs.

UNIX a donné naissance à une \textbf{famille de systèmes} comme BSD, GNU/Linux,
iOS ou encore MacOS, eux-mêmes divisés en variantes de systèmes d'exploitation
aux \textbf{normes POSIX}. Cette norme technique standardise des outils ou des
fonctionnements que l'on devrait retrouver dans un système d'exploitation.

\paragraph{} Il faut savoir que la quasi-totalité des systèmes PC ou mobiles (à
l'exception des systèmes Windows) sont sur des systèmes basés sur la norme POSIX (y compris
ceux de la marque Apple). D'une certaine manière, on peut dire qu'ils sont
descendants ou directement reliés à UNIX.

\subsubsection*{GNU: "GNU's Not UNIX"}\label{subsubsec:gnu}

\paragraph{} GNU est un \textbf{projet des années 1990} lancé par Richard
Stallman. C'est un \textbf{système d'exploitation libre}, bien qu'incomplet,
compatible avec la norme POSIX. GNU est incomplet car il lui manque un noyau
viable; il est donc très souvent utilisé avec le noyau Linux, développé par
Linus Torvalds, ce qui a donné le projet GNU/Linux (c.f. section
\ref{subsubsec:gnulinux}).

Sa \textbf{philosophie} est de maintenir intacts les \textbf{tradictions hackers
de partage} dans un monde de plus en plus marqué par l'empreinte du droit
d'auteur. Il se bat donc pour une \textbf{libre diffusion des connaissances}.
"GNU vise à ne laisser l'homme devenir ni l'esclave de la machine ni de ceux
qui auraient l'exclusivité de sa programmation".

\paragraph{} Il est \textbf{utilisable et partageable librement par tous},
ainsi chacun complète petit a petit l'architecture initiale de GNU pour le
rendre meilleur. C'est dans ce contexte que le projet invite la communauté de
hackers à le rejoindre et à participer à son développement. Il faut savoir
qu'il est composé :

\begin{itemize}
	\item d'un éditeur de texte (emacs),
	\item d'un compilateur très performant (gcc),
	\item d'un débogueur (gdb),
	\item d'un langage de script (bash),
	\item de bibliothèques de systèmes (glibc),
\end{itemize}

\subsubsection*{GNU/Linux}\label{subsubsec:gnulinux}

\paragraph{} GNU/Linux a été créé en \textbf{1991}. Il est \textbf{initié par
le projet Debian} et la naissance du noyau Linux. Il crédite donc à la fois
Linux et GNU \textit{mais l'usage de Linux est plus connu au grand public}.

Il est alors toujours basé sur le \textbf{mouvement du logiciel libre et du
mode opératoire du hacker}. Cette association a eu lieu pour combler le vide
causé par le développement inachevé de GNU (c.f. section \ref{subsubsec:gnu}).

\paragraph{} Il est utilisé sur la plupart des téléphones portables (par
exemple Android) comme sur les super-ordinateurs. Ce projet eut un grand impact
dans le monde des serveurs informatiques. GNU/Linux veut casser le fait qu'à
l'origine il fallait des connaissances solides en informatique pour utiliser un
système d'exploitation (\textit{pas d'interface graphique et besoin d'installer
toutes les applications soi-même}).

Il a donc été un important vecteur de \textbf{popularisation du mouvement de
l'open source}. Il a eu des \textbf{centaines de milliers de redistributions}
avec des versions différentes pour plaire à tous les goûts (en fonction des
besoins, configurations, sécurité, ...\textit{voir différents OS}).

GNU/Linux a remplacé d'autres systèmes de type UNIX et/ou évite l'achat d'une
licence Windows (qui est très chère à l'achat).

\paragraph{} Aujourd'hui on peut retrouver tous les équivalents des
logiciels/applications qu'il y a sous Windows mais en Open Source.
