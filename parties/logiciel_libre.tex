Tout logiciel libre est défini par la \textbf{licence publique générale GNU} (\textit{appelé GPL})
 qui rend les logiciels indépendants de tous éditeurs en les encourageant à l'entraide et au partage.

La licence générale est définie par 4 lois principales :
\begin{itemize}
	\item utiliser un logiciel sans restrictions,
	\item étudier le logiciel,
	\item modifier pour l'adapter aux besoins des utilisateurs,
	\item redistribuer sous certaines conditions précises.
\end{itemize}
Cela a beaucoup d'avantages comme \textbf{correction rapide de bogues et de failles de sécurité}.

\textbf{Attention}, un logiciel libre n'est pas nécessairement gratuit et inversement un logiciel gratuit n'est pas forcement libre ! \newpage
