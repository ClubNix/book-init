Tout logiciel libre est défini par la \textbf{licence publique générale GNU}
 (\textit{appelé GPL}) qui rend les logiciels indépendants de tous éditeurs en
 les encourageant à l'entraide et au partage.

La licence générale est définie par 4 lois principales :
\begin{itemize}
	\item pouvoir utiliser un logiciel sans restrictions,
	\item pouvoir étudier le logiciel,
	\item pouvoir modifier pour l'adapter aux besoins des utilisateurs,
	\item pouvoir redistribuer sous certaines conditions précises.
\end{itemize}
Cela a beaucoup d'avantages comme la possibilité de \textbf{corrigé rapidement
des bogues et des failles de sécurité}.\newline

\textbf{Attention}, un logiciel libre n'est pas nécessairement gratuit et
inversement un logiciel gratuit n'est pas forcement libre !
