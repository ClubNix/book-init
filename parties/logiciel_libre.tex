Tout logiciel libre est defini par la \textbf{licence publique generale GNU} (\textit{appelé GPL}) qui rend les logiciels independants de tout editeurs en les encourage a l'entraide et le partage.

La licence generale est defini par 4 lois principales :
\begin{itemize}
	\item utiliser un logiciel sans restrictions,
	\item etudier le logiciel,
	\item modifier pour l'adapter aux besoins des utilisateurs,
	\item reditribuer sous certaines conditions precises.
\end{itemize}
Cela a bcp d'avantages comme \textbf{correction rapide de bogues et de failles de securites}.

\textbf{Attention}, un logiciel libre n'est pas necessairement gratuit et inversement un logiciel gratuit n'est pas forcement libre ! \newpage
