\paragraph{} Un logiciel libre est un logiciel qui est distribué sous une
``licence libre'' et qui fournit son code source en accord avec dite licence.

Une licence considérée comme ``licence libre'' doit définir ces 4
règles principales:
\begin{itemize}
	\item Permission d'utiliser le logiciel sans restrictions
	\item Permission d'étudier le logiciel et son code source
	\item Permission de le modifier, ce qui permet de l'adapter aux besoins des
		utilisateurs
	\item Permission de redistribuer, peut-être sous certaines conditions
		précises
\end{itemize}
Cela a beaucoup d'avantages comme la possibilité de \textbf{corriger rapidement
des bogues et des failles de sécurité}.

\paragraph{} L'initiateur du mouvement du logiciel libre est le projet GNU.
Ainsi une des licences préfaites les plus utilisées est la \textbf{licence
publique générale GNU} (appelé \textbf{GPL}). Il existe cependant beaucoup
d'autres licences, plus ou moins permissives sur d'autres points.

\paragraph{Attention:} un logiciel libre n'est pas nécessairement gratuit et
inversement un logiciel gratuit n'est pas forcement libre!
