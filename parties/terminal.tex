\subsubsection{Quoi et pourquoi ?}
Le terminal est un programme lancant une console permettant d'executer des commandes. Les commandes permettent en une ligne de texte d'effectuer des opérations qui peuvent s'averer tres longues avec l'interface graphique. Par exemple, modifier les droits d'accès ou d'écriture à un programme s'effectue en une dizaine de clics avec l'interface graphique, alors que la commande *chmod* avec les droits voulu et le nom du fichier le fait instantanément. 

\subsubsection{Ouvrir une console et s'en servir}
Pour ouvrir une console de terminal, on peut :

\begin{itemize}
\item chercher terminal dans la barre de recherche;
\item avec le raccourcis clavier disponible sur la plupart des environnements de bureau avec Ctrl + Alt + T.
\end{itemize}

Une première ligne apparait, et est comme ceci :

\begin{itemize}
\item \textit{nom d'utilisateur"@"nom du pc":\~\$}
\end{itemize}

Tappez alors votre ligne de commande puis \textit{Enter} pour l'executer
Il existe de nombreux outils dans le terminal, que nous allons voir ici :
\paragraph{Arrêter une commande}

Il est possible de lancer une commande puis de l'arreter manuellement sans attendre qu'elle se termine. Par exemple, vous avez lancé une commade ping pour tester votre réseau. La commande ping ne s'arrete que si on lui dit. On peut alors tapper la commande suivante \textit{Ctrl + C} pour l'arreter. Attention cependnat. Meme si sur la commande ping l'arret de la commande n'a pas d'impact, ce n'est pas le cas pour toutes les commandes.

\paragraph{copier-coller}

Copier-coller une ligne de commande depuis un forum est possible, mais pas avec les raccourcis claviers classique. En effet, Ctrl + C est deja une commande du terminal. Il faut donc faire \textit{Ctrl + Shift + C} pour copier une ligne et \textit{Ctrl + Shift + V} pour coller. Vous pouvez aussi, pour les ordinateurs en disposant, séléctionner la ligne "en bleu" et cliquer sur la molette de la souris afin de coller la ligne dans le terminal, ou en utilisant la bouton central au dessus du pavé tactile. 
Il faut faire attention toute fois avec le copier coller depuis les forum. En effet, si vous copiez coller une suite de commande avec des retours à la lignes comme par exemple :

\begin{itemize}
\item \textit{ls}
\item \textit{cd dossier}
\item \textit{cat fichier}
\end{itemize}

Le terminal executera les deux premieres commandes car elles sont séparées par un retour à la ligne. Cela peut être très pratique pais aussi dangereux.
\subsubsection{L'auto-complétion}
Certaines lignes peuvent etre longues a tapper. Le terminal pet à disposition une touche permettant de compléter seul la fin de la commande. c'est la touche Tab. Apres avoir tapper 3 lettres, vous pouvez demander l'auto-complétation. C'est le cas par exemple pour un nom de fichier tres long ou de paquets. Il suiffit alors de tapper "cd début + Tab " et le terminal finira à votre place. Lorsque plusieurs fichiers ont le meme début de nom, le terminal vous les proposera alors en dessous de votre ligne de commande.

\paragraph{}
\textbf{Le manuel}
La plupart des commandes disposent d'un manuel, qui renseigne sur les paramètres de la commande, son utilité, ou encore comment l'utiliser. Pour ouvrir le manuel d'une commande, on tappe dans le terminal \textit{man + 'nom de la commande'}.

\paragraph{}
\textbf{Retrouver une commande déjà tapée précédement}
Pour retrouver une commande déjà tappée, on peut cliquer sur la fleche du haut. Un clic remonte d'une commande. De ce fait, si vous souhaitez tapper une commande très longue et que vous avez déjà tappé il y a quelques temps, cliquez sur la flèche du haut autant de fois que nécéssaire pour la retrouver.
Il existe aussi une commande, plus lourde, \textit{history} qui affiche les 500 dernières commandes tapées.

\subsubsection{Les droits SUDO}
Pour exécuter certaines commandes, notement installer des paquets ou reboot la machine, le terminal à besoin de certains droits, un mot de passe.  C'est les droits SUDO, pour Super Utilisateur DO. 
Les droits sudo concernent les commandes administrateurs systèmes. Quiconque qui détiendrait les droits sudo pourrait passer des commandes de bas niveau capables de modifier gravement la configuration même, donc influer sur le comportement de la machine. En accordant les droits sudo à une commande, elle est alors capable d'installer un programme ( \textit{sudo apt install nom\_ du\_ programme}), modifier un fichier de configuration ... Pour lancer une comande administrateur système, il faut tapper \textit{sudo + nom\_ de\_ la\_ commande}. Le terminal va alors vous demander votre mot de passe administrateur avant de lancer la commande. Si le mot de passe ne s'affiche pas, ni meme des astérix ou autres, c'est normal, c'st pour renfocer la sécurité car personne ne saura ne serait-ce que la taille de ce mot de passe. Attention, vous etes le seul responsable de votre machine et lancer des commandes sudo pourraient complétement détruir votre machine.

\subsubsection{Les questions des commandes}
Certaines commandes vous posent des questions, comme par exemple lors de l'installation d'un paquet
ces questions sont de la forme :

\begin{itemize}
\item 'question' [Y/N]. Il faut alors tapper Y (pour yes) ou N (pour no) puis entrer afin de répondre à la question.
\item 'question' [Y/n]" ou "'question' [y/N]" C'est une variante dans laquelle vous pouvez toujours tapper y ou n mais aussi directement entrée. La réponse prise en compte sera celle en majuscule. 
\end{itemize}

\textit{il existe un poly regroupant les principales commandes terminal disponible sur le git du club}
