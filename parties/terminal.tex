\subsubsection{Quoi et pourquoi ?}
Le terminal est un programme lançant une console permettant d'exécuter des commandes.
 Les commandes permettent en une ligne de texte d'effectuer des opérations qui peuvent
 s'avérer très longues avec l'interface graphique. Par exemple, modifier les droits d'accès ou d'écriture à un programme
 s'effectue en une dizaine de clics avec l'interface graphique, alors que la commande *chmod* avec les droits voulus et
 le nom du fichier le fait instantanément.

\subsubsection{Ouvrir une console et s'en servir}
Pour ouvrir une console de terminal, on peut :

\begin{itemize}
\item chercher terminal dans la barre de recherche;
\item avec le raccourci clavier disponible sur la plupart des environnements de bureau avec Ctrl + Alt + T.
\end{itemize}

Une première ligne apparaît, et est comme ceci :

\begin{itemize}
\item \textit{nom d'utilisateur"@"nom du pc":\~\$}
\end{itemize}

Tapez alors votre ligne de commande puis \textit{Enter} pour l'exécuter
Il existe de nombreux outils dans le terminal, que nous allons voir ici :
\paragraph{Arrêter une commande}

Il est possible de lancer une commande puis de l'arrêter manuellement sans attendre qu'elle se termine.
 Par exemple, vous avez lancé une commande ping pour tester votre réseau. La commande ping ne s'arrête que si on lui dit.
  On peut alors taper la commande suivante \textit{Ctrl + C} pour l'arrêter. Attention cependant.
   Même si sur la commande ping l'arrêt de la commande n'a pas d'impact, ce n'est pas le cas pour toutes les commandes.

\paragraph{copier-coller}

Copier-coller une ligne de commande depuis un forum est possible, mais pas avec les raccourcis claviers classiques.
 En effet, Ctrl + C est déjà une commande du terminal. Il faut donc faire \textit{Ctrl + Shift + C}
 pour copier une ligne et \textit{Ctrl + Shift + V} pour coller. Vous pouvez aussi, pour les ordinateurs en disposant,
  sélectionner la ligne "en bleu" et cliquer sur la molette de la souris afin de coller la ligne dans le terminal,
  ou en utilisant le bouton central au dessus du pavé tactile.
Il faut faire attention toutefois avec le copier coller depuis les forums. En effet, si vous copiez collez une suite de
 commandes avec des retours à la ligne comme par exemple :

\begin{itemize}
\item \textit{ls}
\item \textit{cd dossier}
\item \textit{cat fichier}
\end{itemize}

Le terminal exécutera les deux premières commandes car elles sont séparées par un retour à la ligne. Cela peut être très pratique mais aussi dangereux.
\subsubsection{L'auto-complétion}
Certaines lignes peuvent être longues à taper. Le terminal met à disposition une touche permettant de compléter seul la fin de
la commande. C'est la touche Tab. Après avoir tapé 3 lettres, vous pouvez demander l'auto-complétion.
C'est le cas par exemple pour un nom de fichier très long ou de paquets. Il suffit alors de taper "cd début + Tab " et
le terminal finira à votre place. Lorsque plusieurs fichiers ont le même début de nom, le terminal vous les proposera alors
en dessous de votre ligne de commande.

\paragraph{}
\textbf{Le manuel}
La plupart des commandes disposent d'un manuel, qui renseigne sur les paramètres de la commande, son utilité,
ou encore comment l'utiliser. Pour ouvrir le manuel d'une commande, on tape dans le terminal \textit{man + 'nom de la commande'}.

\paragraph{}
\textbf{Retrouver une commande déjà tapée précédemment}
Pour retrouver une commande déjà tapée, on peut cliquer sur la flèche du haut. Un clic remonte d'une commande.
 De ce fait, si vous souhaitez taper une commande très longue et que vous avez déjà tapée il y a quelques temps,
 cliquez sur la flèche du haut autant de fois que nécessaire pour la retrouver.
Il existe aussi une commande, plus lourde, \textit{history} qui affiche les 500 dernières commandes tapées.

\subsubsection{Les droits SUDO}
Pour exécuter certaines commandes, notamment installer des paquets ou reboot la machine, le terminal a besoin de certains droits,
 un mot de passe.  C'est les droits SUDO, pour Super Utilisateur DO.
Les droits sudo concernent les commandes administrateurs systèmes. Quiconque qui détiendrait les droits sudo pourrait passer des
commandes de bas niveau capables de modifier gravement la configuration même, donc influer sur le comportement de la machine.
En accordant les droits sudo à une commande, elle est alors capable d'installer un programme
( \textit{sudo apt install nom\_ du\_ programme}), modifier un fichier de configuration ...
Pour lancer une commande administrateur système, il faut taper \textit{sudo + nom\_ de\_ la\_ commande}.
Le terminal va alors vous demander votre mot de passe administrateur avant de lancer la commande.
Si le mot de passe ne s'affiche pas, ni même des astérisques ou autre, c'est normal, c'est pour renforcer la sécurité
car personne ne saura ne serait-ce que la taille de ce mot de passe. Attention, vous êtes le seul responsable de votre machine
et lancer des commandes sudo pourraient complètement détruire votre machine.

\subsubsection{Les questions des commandes}
Certaines commandes vous posent des questions, comme par exemple lors de l'installation d'un paquet
ces questions sont de la forme :

\begin{itemize}
\item 'question' [Y/N]. Il faut alors taper Y (pour yes) ou N (pour no) puis entrer afin de répondre à la question.
\item 'question' [Y/n]" ou "'question' [y/N]" C'est une variante dans laquelle vous pouvez toujours taper y ou n
mais aussi directement entrée. La réponse prise en compte sera celle en majuscule.
\end{itemize}

\textit{il existe un poly regroupant les principales commandes terminal disponibles sur le git du club}
