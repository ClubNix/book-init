\subsubsection{Le LDAP, ou \textit{Lightweigth Directory Acess Protocol}}

Lorsqu'un utilisateur tente de se connecter à un ordinateur, le LDAP reçoit une combinaison login/mot de passe et donne
ensuite ou non l'autorisation de se connecter.


\subsubsection{Introduction}

Le LDAP est un protocole créé en 1995, succédant au protocole DAP et permettant l'accès et la modification de base de données
 sur les utilisateurs d'un reseau. Un LDAP sert notement à se connecter ou se déconnecter d'un serveur hébergant le LDAP afin
  d'y être identifié, mais aussi à chercher des informations, les comparer, ajouter des utilisateurs...
\subsection{Le LDAP du club Nix ou de l'école}
Le LDAP du Club*Nix ou de l'école permet de se connecter et de s'identifier afin de récupérer ses informations, quelle que soit
 la machine utilisée.


\subsubsection{Les entrées}

Un LDAP définit l'accès aux entrées (ou utilisateurs la plupart du temps).
Le LDAP ne peut gérer que des entrées. Une entrée peut être un nom d'utilisateur, un périphérique ou encore des paramètres.
 Il existe 2 types d'entrées, les entrées normales et les entrées opérationnelles :
\begin{itemize}
    \item Les entrées classiques, telles que le nom d'utilisateur ou la date d'anniversaire sont des entrées dites classiques.
    \item Les entrées opérationnelles, tels que les paramètres, les dates de modification, qui ne sont accessibles et utilisables
    uniquement par le serveur.
\end{itemize}
Une entrée est définie par son nom, ou DN pour \textit{Distinguished Name}, composé d'une série de clés et de valeurs de ces clés.
Par exemple, la clé \textit{uid} définit le nom d'utilisateur et la clé \textit{cn} définit le nom. Pour ces deux clés,
 une entrée serait sous la forme : \textit{uid=utilisateur,cn=nix}


\subsubsection{L'arborescence}

Un serveur hébergeant un LDAP est organisé selon une arborescence, comme un système de fichiers, dans lequel chaque branche
correspond à une entrée. Une branche située à la racine sera appelée racine ou root en anglais.
Le schéma d'une clé correspond à l'ensemble des valeurs des attributs ou valeurs attribuées aux clés.
Les annuaires LDAP répondent à certaines règles de structure :
\begin{itemize}
    \item Un annuaire est un arbre d'entrées;
    \item Une entrée est constituée d'un ensemble d'attributs;
    \item Un attribut possède un nom, un type et une ou plusieurs valeurs;
    \item Les attributs sont définis dans des schémas.
\end{itemize}


\subsubsection{Chercher des informations d'un LDAP}

Le protocole LDAP fournit un ensemble de fonctions permettant d'interroger le
serveur sur lequel est hébergé le serveur LDAP afin de modifier, ajouter ou
supprimer des entrées. On peut citer notamment *add* pour ajouter une entrée,
 *delete* pour la supprimer, ou *rename* pour la renommer, afin de modifier
 l'arborescence du LDAP.

On utilise essentiellement les LDAP grâce à des interfaces graphiques ou à
d'autres logiciels qui utilisent un LDAP.
