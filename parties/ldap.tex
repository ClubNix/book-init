\subsubsection{Le LDAP, ou \textit{Lightweigth Directory Acess Protocol}}

\subsubsection{Introduction}

Le LDAP est un protocole créé en 1995, succédant au protocole DAP et est une
base de données sous forme d'arbre. Les serveurs LDAP sont principalement
utilisés dans les grandes entreprises ou écoles/universités pour enregistrer
les informations sur les employés/élèves, et vérifier leurs identifiants sur
différents services.

\paragraph{} Par exemple, une entreprise peut connecter leurs ordinateurs
GNU/Linux, Windows, etc. et leurs services web à un serveur LDAP pour que les
employés n'aient qu'un seul identifiant et un seul mot de passe sur tous ces
services.

\subsection{Le LDAP du club Nix ou de l'école}

Le Club*Nix et l'école possèdent chacun un serveur LDAP contenant les
informations des comptes des membres/élèves, leur permettant entre autres de se
connecter aux ordinateurs avec le même mot de passe.

\subsubsection{Les entrées}

\paragraph{} Une base de donnée de type LDAP contient des entrées (utilisateurs la plupart
du temps), qui sont des collections d'attributs. Un attribut peut être un nom
d'utilisateur, un périphérique ou autres.

\paragraph{} Comme le LDAP est sous forme arborescente (comme un système de
fichier), chaque entrée peut être accédée via son ``chemin'', ici appelé DN
pour \textit{Distinguished Name}. Contrairement à un
système de fichier, le chemin se lit de droite à gauche (au lieu de gauche à
droite), et chaque ``dossier'' est sous la forme ``\texttt{clé=valeur}''.

\paragraph{} Ainsi un exemple de DN d'une entrée serait:
\texttt{uid=john,ou=Membres,dc=clubnix,dc=org}

\paragraph{} Dans cet exemple, en faisant le parallèle avec un système de
fichier, ici la racine est ``\texttt{dc=clubnix,dc=org}'' (un peu comme le
\texttt{C:} dans un système de fichier Windows), ensuite dans le ``dossier''
``\texttt{ou=Membres}'', enfin dans le ``fichier'' (ou ici l'entrée)
``uid=john''.

\subsubsection{Chercher des informations d'un LDAP}

\paragraph{} Le protocole LDAP fournit un ensemble de fonctions permettant
d'interroger le serveur sur lequel est hébergé le serveur LDAP afin de
modifier, ajouter ou supprimer des entrées. On peut citer notamment *add* pour
ajouter une entrée, *delete* pour la supprimer, ou *rename* pour la renommer,
afin de modifier l'arborescence du LDAP.

\paragraph{} On utilise essentiellement les LDAP grâce à des interfaces
graphiques ou à d'autres logiciels qui utilisent un LDAP, mais des outils en
ligne de commande, tels que \texttt{ldapadd} ou \texttt{ldapsearch} peuvent
aussi être utilisés.
