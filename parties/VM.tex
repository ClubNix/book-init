\subsubsection{Définitions}

\paragraph{La Virtualisation}
 Au sens large, la virtualisation consiste à simuler l'existence de plusieurs machines informatiques en utilisant une seule.
 Ceci permet en particulier de diminuer les coûts d'achat de matériel informatique et de rentabiliser leur utilisation.
Exemples de logiciels de virtualisation: VMware, VirtualBox...

\paragraph{Un Hyperviseur}
\begin{itemize}
	\item assure le contrôle du processeur et des ressources de la machine hôte
	\item Alloue à chaque machine virtuelle (VM) les ressources dont elle a besoin
	\item S'assure que ces VM n'interfèrent pas l'une avec l'autre
\end{itemize}

Il existe deux types d'hyperviseurs: le type 1 et le type 2

\paragraph{Définition d'une machine virtuelle}

Une machine virtuelle est un fichier informatique, généralement appelé image, qui se comporte comme un ordinateur réel.
En d’autres termes, il s’agit d’un ordinateur créé à l’intérieur d’un ordinateur. Elle s’exécute dans une fenêtre,
 comme tout autre programme, en offrant à l’utilisateur final une expérience identique à celle qu’il aurait sur le
 système d’exploitation hôte. La machine virtuelle est placée dans un « bac à sable » qui l’isole du reste du système,
 de sorte que les logiciels installés sur la machine virtuelle ne peuvent ni s’échapper, ni modifier l’ordinateur hôte.
 Cela produit un environnement idéal pour tester d’autres systèmes d’exploitation, dont des versions bêta,
 l’accès à des données infectées par des virus, la création de sauvegardes de système d’exploitation et l’exécution
 de logiciels ou d’applications sur des systèmes d’exploitation auxquels ils ne sont pas destinés à l’origine.
Il est possible d’exécuter plusieurs machines virtuelles simultanément sur un même ordinateur physique.
Pour les serveurs, les divers systèmes d’exploitation fonctionnent côte à côte, avec un composant logiciel appelé hyperviseur
(logicielle de virtualisation) pour les gérer, alors que les ordinateurs de bureau classiques n’utilisent qu’un seul
système d’exploitation pour exécuter d’autres systèmes d’exploitation dans des fenêtres de programme qui leur sont propres.
 Chaque machine virtuelle fournit son propre matériel virtuel, à savoir les processeurs, la mémoire, les disques durs,
  les interfaces réseau et les autres périphériques nécessaires. Le matériel virtuel est ensuite mappé au matériel réel
   sur la machine physique, ce qui permet de réaliser des économies en réduisant le besoin de disposer de systèmes matériels physiques,
    ainsi que les coûts de maintenance associés, tout en réduisant la demande en alimentation et refroidissement.


VirtualBox est une application de virtualisation x86 qui facilite la création de machines virtuelles mais pas très conseillé
d’utilisation, il est préférable d'utiliser Docker(lxd) qui est très léger et permet d'éviter une surcharge
Sinon nous avons KVM qui lui est un peu plus lourd mais va plutôt se rapprocher de virtual Box.
