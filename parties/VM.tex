\subsubsection{Définitions}

\paragraph{La Virtualisation} Au sens large, la virtualisation consiste à
simuler l'existence de machines informatiques dans une machine informatique.
Ceci permet en particulier de diminuer les coûts d'achat de matériel
informatique et de rentabiliser leur utilisation. Exemples de logiciels de
virtualisation: KVM, Xen, VMware, VirtualBox\ldots

\paragraph{Un Hyperviseur}
\begin{itemize}
	\item assure le contrôle du processeur et des ressources de la machine hôte
	\item Alloue à chaque machine virtuelle (VM) les ressources dont elle a besoin
	\item S'assure que ces VM n'interfèrent pas l'une avec l'autre
\end{itemize}

\paragraph{Une machine virtuelle}

\paragraph{} Une machine virtuelle est généralement enregistré sous la forme
d'un fichier, généralement appelé image, qui se comporte comme un ordinateur
réel.

\paragraph{} Cette image contient les informations des partitions de la machine
et le contenu des partitions.

\paragraph{} Lorsque la machine virtuelle est lancée, elle est placée dans un
``bac à sable'' qui l’isole du reste du système, de sorte que les programmes de
la machine virtuelle ne peuvent ni s’échapper, ni modifier l'ordinateur hôte.
Cela produit un environnement idéal pour tester d'autres systèmes
d’exploitation, dont des versions bêta, l’accès à des données infectées par des
virus.

\paragraph{} Cependant, leur principale utilisation en entreprise est pour des
raisons de sécurité: plusieurs services sont hébergés sur plusieurs machine
virtuelles. Ainsi s'il s'avère qu'un système est vulnérable, un attaquant aura
beaucoup de mal à sortir du système de virtualisation.
