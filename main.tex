\documentclass[a4paper]{report}

%% Language and font encodings
\usepackage[french]{babel}
\usepackage[utf8x]{inputenc}
\usepackage[T1]{fontenc}

%% Sets page size and margins
\usepackage[a4paper,top=3cm,bottom=2cm,left=3cm,right=3cm,marginparwidth=1.75cm]{geometry}

%% Useful packages
\usepackage{amsmath}
\usepackage{graphicx}
\usepackage[colorinlistoftodos]{todonotes}

\title{Book Initiation}
\author{Club*NIX}

\begin{document}
\maketitle

\tableofcontents
\newpage


\chapter{Linux}
 \section{GNU-Linux}
    \subsubsection{UNIX}

\textbf{Unix} est un \textbf{systeme d'exploitation} multi-taches et multi-utilisateurs créé en \textbf{1969}. Il repose sur un interpretateur (le shell) et sur plusieurs petits utilisateurs (avec des actions specifiques defini lors du code). C'est une \textbf{marque deposée de l'OpenGroup}
Son nom est un derivé de "Unics" (\textit{Uniplexed Information and Computing Service}). C'est un jeu de mot avec "Multics" (un autre noyau) qui vise a offrir simultanement plusieurs services a un ensemble d'utilisateur.
Unix a donner naissance a une \textbf{famille de systemes} comme BSD, GNU/LINUX, iOS ou encore MacOS, eux-même divisé en variante de système d'exploitation aux \textbf{normes POSIX}. Cette norme technique standardise les interfaces de programmation des logiciels. 
Il faut savoir que la casi totalité des systèmes pc ou mobiles (à l'exception des Windows) sont basés sur le noyau de UNIX (y compris Apple). 
Comme dit precedement, Unix est un noyau, \textbf{il doit donc etre utilisé avec d'autres logiciels} car il lui faut un systeme d'exploitation. \newline


\subsubsection{GNU : "GNU's Not UNIX"}

GNU est un \textbf{projet des annees 1990} lancé par Richard STALLMAN. C'est un \textbf{système d'exploitation libre} compatible avec le noyau UNIX. Il utilise des logiciels libres issus d'autres projets tels que le noyaux Linux (\textit{voir GNU/LINUX}). 
Sa \textbf{philosophie} est de maintenir intacte les \textbf{tradictions hackers de partage}] dans un monde de plus en plus marqué par l'empreinte du droit d'auteur. Il se bat donc pour une \textbf{libre diffusion des connaissances}. "GNU vise à ne laisser l'homme devenir ni l'esclave de la machine ni de ceux qui auraient l'exclusivité de sa programmation". 
Il est \textbf{utilisable et partagable librement par tous}, ainsi chacun complete petit a petit l'architecture initale de GNU pour le performer. C'est dans ce contexte que le projet invite la communauté de hacker a le rejoindre et a participer a son developement. 
Il faut savoir qu'il est composé : 
\begin{itemize}
	\item d'un editeur de texte (emacs),
	\item d'un compilateur tres performant (gcc),
	\item d'un debogueur (gdb), 
	\item d'un langage de script (bash),
	\item des bibliotheques de systemes (glibc),  
	\item (plus tard noyau ramener par un projet linux ). 
\end{itemize}
Malgres tout ces composants, GNU est tjrs \textbf{incomplet} : son noyau est immature et imcompatible avec certains de ses composants. \newline


\subsubsection{GNU/LINUX}

GNU/LINUX a été créé en \textbf{1991}. Il est \textbf{initié par le projet Debian} et la naissance du noyau Linux. Il credite donc a la fois Linux et GNU \textit{mais l'usage de Linux est plus connus au grand publique}. Il est alors toujours baser sur le \textbf{mouvement du logiciel libre et du mode operatoire du hacker}. 
Cet associement a eu lieu pour combler le vide causé par le développement inachevé de GNU (\textit{voir GNU}). Il est utilisé sur la plupart des telepones portables comme sur les super-ordinateurs (par exemple, android équipe 80\% de ses smartphones).Ce projet fut un grand d'impacte dans le monde des serveurs informatiques.
GNU/LINUX veut casser le faite qu'a l'origine il fallait des connaissances solide en informatiques pour utiliser un système d'exploitation (\textit{pas d'interface graphique et besoin d'installer toutes les applications soit même}).Il a donc été un important vecteur de \textbf{popularisation du mouvement de l'open source}.
Il a eu des \textbf{centaines de miliers de redistributions} avec des versions differentes pour plaire a tout les gouts (en fonction des besoins, configration, securite, ...\textit{voir differents OS}). GNU/LINUX a remplacé d'autres systemes de type Unix et/ou evite l'achat de licence Windows (qui est trés cher à l'achat). Aujourd'hui on peut retrouver tout les equivalent des logiciel/applications qu'il y a sous Windows mais en Open Sources.\newline

 \section{different OS}
    \paragraph{} En informatique, un système d'exploitation est un ensemble de
programmes qui dirige l'utilisation des capacités d'un ordinateur par des
logiciels applicatifs.

\subsubsection{Différents types d'OS basés sur différents noyaux}

\begin{description}
	\item[MacOS:] série d'interfaces graphiques basée sur l'opération des
		systèmes développé par Apple pour leur Macintosh, originellement basé
		sur un système UNIX
	\item[iOS:] système d'exploitation pour mobile développé et distribué par
		Apple pour iPhone et iPod
	\item[GNU/Linux:] système d'exploitation pour ordinateur
		assemblé sous le modèle du "Free and Open Source Software" ou
		\textit{FOSS}
	\item[Android:] c'est un dérivé de GNU/Linux ; système d'opération désigné
		principalement pour les mobiles tactiles comme les smartphones
		et les tablettes, initialement développé par Android
	\item[Microsoft Windows:] série d'interfaces graphiques développée et
		commercialisée par Microsoft, basé originellement sur MS-DOS qui était
		développé par IBM
	\item[BSD:] réputé pour réhabiliter le rôle des serveurs, organisé par
		les programmeurs d'UNIX
\end{description}


\subsubsection{Différentes familles d'OS Linux}

\begin{description}
	\item[Debian/Ubuntu:] \textit{1993}
		\begin{itemize}
			\item développé par SPI (\textit{Software in the Public Interest})
			\item caractère non commercial et mode de gouvernance coopératif
			\item déjà installé avec son noyau, ses pilotes, son programme
				d'installation de distribution, ses logiciels "utiles" (pour le
				WiFi, un navigateur web, etc.)
			\item réunit une dizaine de sous-familles Debian avec
				principalement les mêmes composants mais avec des différences
				plus ou moins mineures. Les plus connues sont: KaliLinux,
				Kubuntu, Raspbian, Ubuntu, Xubuntu, etc.
		\end{itemize}

	\item[Red Hat/Cent OS/Fedora:] \textit{1994}
		\begin{itemize}
			\item développé et distribué par l'entreprise Red Hat (entreprise,
				rachetée par IBM en 2018, dédiée aux logiciels Open source +
				distributeur de système d'exploitation GNU/Linux)
			\item plusieurs distributions importantes en sont issues: Fedora,
				Enigma, etc.
			\item principalement destinées aux serveurs des entreprises
			\item Fedora voulait faire passer progressivement les utilisateurs
				Windows sous GNU/Linux
		\end{itemize}

	\item[Arch:] \textit{2002}
		\begin{itemize}
			\item accent sur la simplicité et légèreté et les utilisateurs
				avancés
			\item contribution ouverte (Open Source) tant que cela respecte sa
				philosophie
		\end{itemize}

	\item[Suse/OpenSuse:] \textit{1994}
		\begin{itemize}
			\item distribution communautaire et commerciale
			\item destinée à l'utilisation en entreprise mais toujours Open
				Source
			\item cycle de développement long mais cycle de vie long
			\item disponible à la vente (licence et mise à jour)
		\end{itemize}
\end{description}

    
\chapter{Language}
  \section{LaTeX}
    \subsubsection{Histoire et Principe}

LaTeX a été créé en 1983, c'est un \textbf{langage et un système de composition de documents}.
Il sert principalement à de la mise en page simple de documents et a pour but de séparer le fond et la forme. 

LaTeX est devenu le langage privilégié pour les documents scientifiques du faite de sa simplicité. 

Pour rédigé du LaTeX, il faut uniquement se concentrer sur la structure logique du document et son contenu car le logiciel s'occupe de la mise en page automatiquement. 
Vous pouvez écrire en LaTeX grâce a diffèrent logiciel d’éditeur de texte comme TexMaker, Texlive, TeXworks, TexMacs, ... où il est possible de voir directement la mise en page en format PDF lorsqu'il es compiler.
Mais il est également possible de manipuler LaTeX simplement a partir d'un terminal ou de gedit.

L'évolution de LaTeX est assurée par la communauté d'utilisateurs qui regroupe des étudiants et des professeurs de mathématiques ou de physiques, comme des musiciens et ingénieurs en informatiques notamment.


\subsubsection{Différence avec Word}

Il faut d'abord savoir que les deux logiciels ne se comporte et ne s'utilise pas de la même façon. 

\begin{itemize}
\item La mise en page (images, figures, légendes, formules mathématiques, dessins, tableaux, ...) sur Word est une rude manipulation qui fais perdre du temps. LaTeX le fait tout seul, mais son interface austère fait "peur" aux debutant. Pourtant il suffit de lui dire quel type de documents on souhaite obtenir pour obtenir quelque chose de lisible et adapté avec les normes éditoriales ;
\item LateX est gratuit, contrairement à Word ;
\item Les formules mathématiques sont simple d'écriture sur LaTeX ;
\item Tout est modifiable et paramétrable avec LaTeX à n'importe quel moment (\textit{si au bout de la 100eme page vous vous rendez compte que vous voulez changer de police par exemple}) ;
\item La gestion de documents longs est intuitive sur LaTeX, contrairement a la complexité sur Word lorsqu'il faut gérer la mise en page identique ;
\item LaTeX peut générer automatiquement des bibliothèques ou table de matières beaucoup plus facilement que sur Word ;
\item Accèder à la création des PDF rapidement sur LaTeX ;
\end{itemize}


\subsubsection{Utilisation}

Le balisage est assez semblable au langage HTML. Il est donc possible de \textbf{créé ou modifier des macro-commandes} afin d'ajouter des raccourci. \textit{Par exemple pour regrouper plusieurs instructions en une.}
Comme ce langage a été créé avant la Unicode, tout caractères peut s'écrire en ASCII. 

Il y a tout d'abord tout un package a faire pour déterminer la langue d’écriture, les marges, polices d’écriture, couleur et taille de l’écriture style de document, ...
Tout ces \textbf{peuvent être trouver déjà fait sur Internet}, vous pouvez donc simplement les modifier.


Voici les principales instructions qui vous servirons : 
\begin{itemize}
\item Partie : \_part{\textit{non_de_la_partie}}
\item Section : \_section{\textit{non_de_la_partie}}
\item Sous-section : \_subsection{\texit{non_de_la_partie}}
\item Paragraphe : \_paragraph{\texit{non_de_la_partie}}
\end{itemize}

\begin{itemize}
\item Saut de ligne : \_newline ou \_\_
\item Liste : commencer par \_begin{itemize} puis pour chaque tiret \_item et conclure par \_end{itemize}
 \textit{NB : pour modifier les puces il suffit de mettre la puces souhaité entre [] apres \_item
	pour des liste numéroté il faut écrire \_begin{enumerate} }
\end{itemize} 

\begin{itemize}
\item très très petite écriture : \_scriptsize
\item très petite écriture : \_footnotesize
\item petite écriture : \_small
\item grande écriture : \_large
\item très grande écriture : \_LARGE
\item très très grande écriture : \_huge
\end{itemize}

\begin{itemize}
\item Gras : \_textbf{}
\item Italique : \_textit{}
\item Penché : \_textsl{}
\item Machine à écrire : \_texttt{}
\item Exposant : \_textsuperscript{}
\item Encadré : \_fbox{}
\item Souligné : \_ul{}
\item Barré : \_st{}
\end{itemize}
différentes polices : bch, cmr, lmr, lmss, pag, pbk, phv, ...

  \section{Markdown}
    \subsubsection{Histoire}

Le markdown est un \textbf{langage de balisage} crée en 2004. Il est facile à manipuler, donc simple à écrire et a lire sans connaître
 les balises.
Le markdown peut être écrit sur n'importe quel éditeur de texte, il suffit lorsque le document est prêt a être enregistré
 de nommer le document puis d'inscrire "\textbf{.md}".


\subsubsection{Instructions}

Les instructions sont \textbf{très simples et peuvent être combinées}.
On va voir quelques instructions de base :

\paragraph{Police d'un texte}
\begin{itemize}
	\item Italique : encadrer le(s) mot(s) désiré(s) par * ou \_
	\item Gras :  le(s) mot(s) désiré(s) par **
	\item Souligner : encadrer le(s) mot(s) désiré(s) par \_\_
	\item Barré : encadrer le(s) mot(s) désiré(s) par \~{}\~{}
	\item souligner les titres : mettre sur une ligne en dessous des = ou -
\end{itemize}


\paragraph{Mise en forme du texte}
\begin{itemize}
	\item commencer un paragraphe : mettre 4 espaces
	\item délimiter un paragraphe : sauter une ligne
	\item retour à la ligne : deux espaces à la fin de phrase
	\item titre : \# (rajouter des \# par sous niveaux de paragraphes)
\end{itemize}


\paragraph{Ajout d'éléments au texte}
\begin{itemize}
	\item bloc de code : encadrer le(s) mot(s) désiré(s) par ```
	\item citation : commencer par \textgreater
	\item liste : commencer par * ou - ou +
	\item liste ordonnée : commencer par 1. 2. 3. ...
	\item cases a cocher : [ ] ou [x]
	\item tableau :
		- délimiter les colonnes par : \textbar
		- délimiter les titres des autres lignes par : :-----:
	\item lien :
		- en hypertexte : l'encadrer en \textless et \textgreater
		- en bouton : [ \textit{nom du bouton} ](\#){.btn }
	\item mage : ![ \textit{texte} ]( \textit{url de l'image} )
\end{itemize}

    
\chapter{Terminal}
  \section{Terminal}
    \subsection{Le Terminal}
\subsubsection{Quoi et pourquoi ?}
Le terminal est un programme lancant une console permettant d'executer des commandes. Les commandes permettent en une ligne de texte d'effectuer des opérations qui peuvent s'averer tres longues avec l'interface graphique. Par exemple, modifier les droits d'accès ou d'écriture à un programme s'effectue en une dizaine de clics avec l'interface graphique, alors que la commande *chmod* avec les droits voulu et le nom du fichier le fait instantanément. 
\subsubsection{Ouvrir une console et s'en servir}
Pour ouvrir une consol de terminal, on peut :
\begin{itemize}
\item chercher terminal dans la barre de recherche;
\item avec le raccourcis clavier disponible sur la plupart des environnements de bureau avec Ctrl + Alt + T.
\end{itemize}
Une première ligne apparait, et est comme ceci :
\begin{itemize}
\item \textit{nom d'utilisateur"@"nom du pc":~\$}
\end{itemize}
Tappez alors votre ligne de commande puis \textit{Enter} pour l'executer
Il existe de nombreux outils dans le terminal, que nous allons voir ici :
\paragraph{
\textbf{Arrêter une commande}
Il est possible de lancer une commande puis de l'arreter manuellement sans attendre qu'elle se termine. Par exemple, vous avez lancé une commade ping pour tester votre réseau. La commande ping ne s'arrete que si on lui dit. On peut alors tapper la commande suivante \textit{Ctrl + C} pour l'arreter. Attention cependnat. Meme si sur la commande ping l'arret de la commande n'a pas d'impact, ce n'est pas le cas pour toutes les commandes.}
\paragraph{
\textbf{copier-coller}
Copier-coller une ligne de commande depuis un forum est possible, mais pas avec les raccourcis claviers classique. En effet, Ctrl + C est deja une commande du terminal. Il faut donc faire \textit{Ctrl + Shift + C} pour copier une ligne et \textit{Ctrl + Shift + V} pour coller. Vous pouvez aussi, pour les ordinateurs en disposant, séléctionner la ligne "en bleu" et cliquer sur la molette de la souris afin de coller la ligne dans le terminal, ou en utilisant la bouton central au dessus du pavé tactile. 
Il faut faire attention toute fois avec le copier coller depuis les forum. En effet, si vous copiez coller une suite de commande avec des retours à la lignes comme par exemple :
\begin{itemize}
\item \textit{ls}
\item \textit{cd dossier}
\item \textit{cat fichier}
\end{itemize}
Le terminale executera les deux premieres commandes car elles sont séparées par un retour à la ligne. Cela peut être très pratique pais aussi dangereux.}
\subsubsubtitle{L'auto-complétation}
Certaines lignes peuvent etre longues a tapper. Le terminal pet à disposition une touche permettant de compléter seul la fin de la commande. c'est la touche Tab. Apres avoir tapper 3 lettres, vous pouvez demander l'auto-complétation. C'est le cas par exemple pour un nom de fichier tres long ou de paquets. Il suiffit alors de tapper "cd début + Tab " et le terminal finira à votre place. Lorsque plusieurs fichiers ont le meme début de nom, le terminal vous les proposera alors en dessous de votre ligne de commande.
\paragraph{
\textbf{Le manuel}
La plupart des commandes disposent d'un manuel, qui renseigne sur les paramètres de la commande, son utilité, ou encore comment l'utiliser. Pour ouvrir le manuel d'une commande, on tappe dans le terminal \textit{man + 'nom de la commande'}.}
\paragraph{
\textbf{Retrouver une commande déjà tapée précédement}
Pour retrouver une commande déjà tappée, on peut cliquer sur la fleche du haut. Un clic remonte d'une commande. De ce fait, si vous souhaitez tapper une commande très longue et que vous avez déjà tappé il y a quelques temps, cliquez sur la flèche du haut autant de fois que nécéssaire pour la retrouver.
Il existe aussi une commande, plus lourde, \textit{history} qui affiche les 500 dernières commandes tapées.}
\subsubsection{Les droits SUDO}
Pour exécuter certaines commandes, notement installer des paquets ou reboot la machine, le terminal à besoin de certains droits, un mot de passe.  C'est les droits SUDO, pour Super Utilisateur DO. 
Les droits sudo concernent les commandes administrateurs systèmes. Quiconque qui détiendrait les droits sudo pourrait passer des commandes de bas niveau capables de modifier gravement la configuration même, donc influer sur le comportement de la machine. En accordant les droits sudo à une commande, elle est alors capable d'installer un programme ( \textit{sudo apt install nom_du_programme}), modifier un fichier de configuration ... Pour lancer une comande administrateur système, il faut tapper \textit{sudo + nom_de_la_commande}. Le terminal va alors vous demander votre mot de passe administrateur avant de lancer la commande. Si le mot de passe ne s'affiche pas, ni meme des astérix ou autres, c'est normal, c'st pour renfocer la sécurité car personne ne saura ne serait-ce que la taille de ce mot de passe. Attention, vous etes le seul responsable de votre machine et lancer des commandes sudo pourraient complétement détruir votre machine.
\subsubsection{Les questions des commandes}
Certaines commandes vous posent des questions, comme par exemple lors de l'installation d'un paquet
ces questions sont de la forme :
\begin{itemize}
\item 'question' [Y/N]. Il faut alors tapper Y (pour yes) ou N (pour no) puis entrer afin de répondre à la question.
\item 'question' [Y/n]" ou "'question' [y/N]" C'est une variante dans laquelle vous pouvez toujours tapper y ou n mais aussi directement entrée. La réponse prise en compte sera celle en majuscule. 

\textit{il existe un poly regroupant les principales commandes terminal disponible sur le git du club}

  \section{Bash}
    \subsubsection{Bash et scripts}

\paragraph{} Le bash est un interpréteur de ligne de commande natif aux
systèmes d'exploitation UNIX et GNU/Linux.

\paragraph{A quoi ça sert?}

\paragraph{} Bash est le programme par défaut sous GNU/Linux qui exécute des
commandes. Il peut être utilisé en mode interactif, comme vu à la partie sur la
ligne de commande; mais il peut aussi être utilisé pour exécuter des scripts.

\paragraph{} Un script est tout simplement un regroupement de lignes de
commandes, qui vont être exécutées lignes par lignes.

\paragraph{L'interprétation d'une ligne de commande}

Chaque interprétation d'une ligne de commande, que ce soit en mode intéractif
ou dans un script, repose sur ce format:

\begin{itemize}
	\item Le premier mot de la ligne est interprété comme le nom de la commande
	\item Chaque mot est séparé par un ou plusieurs caractères de séparation
		(espace, tabulation)
	\item Un retour à la ligne, ou un ``\texttt{;}'' si l'on veut mettre
		plusieurs commandes sur une seule ligne
\end{itemize}

Bash propose différents traitements des commandes:
\begin{description}
	\item[Commandes successives:] \mintinline{bash}{com1 ; com2 ; ... ; comN.}
		Les commandes com1 à comN sont exécutées les unes après les autres
	\item[Commandes simultanées:] \mintinline{bash}{com1 & com2 & ...  & comN}.
		Les commandes com1 à comN sont exécutées simultanéments
\end{description}

\paragraph{} Voici un exemple de script Bash:

\begin{minted}{bash}
#!/bin/bash

echo "Enter username"
read username
echo "Enter password"
read password

if [[ $username == "admin" && $password == "secret" ]]; then
	echo "valid user"
else
	echo "invalid user"
fi
\end{minted}


Pour plus d'informations sur les lignes de commandes bash, il existe un poly sur les commandes du terminal disponible sur le GitHub du club.

  \section{Oh My ZSH}
    \input{ZSH.tex}
    
\chapter{Logiciel libre}

\subsection{Logiciel libre}

Tout logiciel libre est defini par la \textbf{licence publique generale GNU} (\textit{appelé GPL}) qui rend les logiciels independants de tout editeurs en les encourage a l'entraide et le partage.

La licence generale est defini par 4 lois principales :
\begin{itemize}
	\item utiliser un logiciel sans restrictions,
	\item etudier le logiciel,
	\item modifier pour l'adapter aux besoins des utilisateurs,
	\item reditribuer sous certaines conditions precises.
\end{itemize}
Cela a bcp d'avantages comme \textbf{correction rapide de bogues et de failles de securites}.

\textbf{Attention,} un logiciel libre n'est pas necessairement gratuit et inversement un logiciel gratuit n'est pas forcement libre ! \newpage

  \section{Git}
    \subsubsection{Définition}
Git est un logiciel de gestion de version décentralisé (pas forcément besoin d'être en ligne) le plus couramment utilisé dans le monde
(il en existe d'autres comme SVN mais beaucoup moins utilisés), il fut développé en 2005 par Linus Torvalds (créateur du noyau linux).
Git est un outil de versionning, c'est à dire que lorsque plusieurs personnes sont sur un même projet, par exemple ils rédigent un code à plusieurs.
 Un projet constitue plusieurs fichiers amenés à évoluer dans le temps. Git va les prévenir de Qui a modifié le fichier, Quand, Quoi et Pourquoi.
 Cet outil est donc primordial durant la conception d'un projet à plusieurs et même seul
 (lorsque l'on revient sur du code écrit il y a 3 mois on ne sait pas toujours à quoi il correspond).
 Lorsqu'on travaille à plusieurs sur un même projet Git fait une copie et une fois modifié, il rassemble tous les fichiers en un.

Attention, il ne faut pas confondre Git avec Github et Gitlab. Github et Gitlab vont rajouter une interface graphique et vont permettre
 de le partager. De plus il ne faut pas confondre Github et Gitlab, Github est géré par une entreprise privée qui entre temps a été
 rachetée par Microsoft (entreprise vouée à l'échec) et l'autre est opensource.

\subsubsection{Commandes de base}
\begin{itemize}
	\item Récupérer un projet \textit{git clone (adresse repository)}
	\item Prendre en compte un changement (créer un point de sauvegarde) \textit{git commit -m(message)}
	\item Récupérer quelque chose qui aurait changé en ligne \textit{git pull}
	\item Mettre en ligne \textit{git push}
\end{itemize}

  \section{F-Droid}
    \textbf{F-Droid} est un magasin d'application (\textit{au même titre que le
PlayStore ou l'AppleStore}) pour smartphones Android qui met à disposition plus
de \textbf{12 000 applications libres et gratuites}.

Il a été créé en 2010 et est promu par la Free SoftWare Foundation Europe. Son
architecture de sécurité est basée sur le modèle de Debian.
Comme toute construction Open Source, F-Droid est tenu et développé par un grand
nombre de contributeurs faisant partie de la communauté.
Ainsi, chaque personne peut créer sa propre application et la mettre à
disposition gratuitement sur ce magasin. F-Droid assure tout de même un niveau
de sécurité, les applications publiées sont vérifiées.\newline

L’avantage d'utiliser F-Droid plutôt que le magasin d'application par défaut est
principalement la préservation de vos données.
Mais aussi la \textbf{sécurité} de vos applications et le fait qu'il n'est pas
nécessaire de s'identifier pour pouvoir télécharger des nouvelles applications.

Ce magasin facilite la découverte et l'installation de multiples applications.
De plus, contrairement à la plupart des applications, vous n’êtes pas obligé de
faire les mises à jour, et pouvez ainsi garder une ancienne version.

    
\chapter{Infrastructure réseau}
  \section{LDAP}
    \subsubsection{Le LDAP, ou \textit{Lightweigth Directory Acess Protocol}}

Lorsqu'un utilisateur tente de se connecter à un ordinateur, le LDAP reçoit une combinaison login/mot de passe et donne
ensuite ou non l'autorisation de se connecter.


\subsubsection{Introduction}

Le LDAP est un protocole créé en 1995, succédant au protocole DAP et permettant l'accès et la modification de base de données
 sur les utilisateurs d'un reseau. Un LDAP sert notement à se connecter ou se déconnecter d'un serveur hébergant le LDAP afin
  d'y être identifié, mais aussi à chercher des informations, les comparer, ajouter des utilisateurs...
\subsection{Le LDAP du club Nix ou de l'école}
Le LDAP du club NIX ou de l'école permet de se connecter et de s'identifier afin de récupérer ses informations, quelle que soit
 la machine utilisée.


\subsubsection{Les entrées}

Un LDAP définit l'accès aux entrées (ou utilisateurs la plupart du temps).
Le LDAP ne peut gérer que des entrées. Une entrée peut être un nom d'utilisateur, un périphérique ou encore des paramètres.
 Il existe 2 types d'entrées, les entrées normales et les entrées opérationnelles :
\begin{itemize}
    \item Les entrées classiques, telles que le nom d'utilisateur ou la date d'anniversaire sont des entrées dites classiques.
    \item Les entrées opérationnelles, tels que les paramètres, les dates de modification, qui ne sont accessibles et utilisables
    uniquement par le serveur.
\end{itemize}
Une entrée est définie par son nom, ou DN pour \textit{Distinguished Name}, composé d'une série de clés et de valeurs de ces clés.
Par exemple, la clé \textit{uid} définit le nom d'utilisateur et la clé \textit{cn} définit le nom. Pour ces deux clés,
 une entrée serait sous la forme : \textit{uid=utilisateur,cn=nix}


\subsubsection{L'arborescence}

Un serveur hébergeant un LDAP est organisé selon une arborescence, comme un système de fichiers, dans lequel chaque branche
correspond à une entrée. Une branche située à la racine sera appelée racine ou root en anglais.
Le schéma d'une clé correspond à l'ensemble des valeurs des attributs ou valeurs attribuées aux clés.
Les annuaires LDAP répondent à certaines règles de structure :
\begin{itemize}
    \item Un annuaire est un arbre d'entrées;
    \item Une entrée est constituée d'un ensemble d'attributs;
    \item Un attribut possède un nom, un type et une ou plusieurs valeurs;
    \item Les attributs sont définis dans des schémas.
\end{itemize}


\subsubsection{Chercher des informations d'un LDAP}

Le protocole LDAP fournit un ensemble de fonctions permettant d'interroger le serveur sur lequel est hébergé
le serveur LDAP afin de modifier, ajouter ou supprimer des entrées. On peut citer notamment *add* pour ajouter une entrée,
 *delete* pour la supprimer, ou *rename* pour la renommer, afin de modifier l'arborescence du LDAP.
\subsubsection{Les principales commandes du LDAP}
Voici une lise des principales clés d'un LDAP :
\begin{itemize}
    \item \textit{userid} (userid), il s'agit d'un identifiant unique obligatoire;
    \item \textit{cn} (common name), il s'agit du nom de la personne;
    \item \textit{givenname}, il s'agit du prénom de la personne;
    \item \textit{sn} (surname), il s'agit du surnom de la personne;
    \item \textit{o} (organization), il s'agit de l'entreprise de la personne;
    \item \textit{u} (organizational unit), il s'agit du service de l'entreprise dans laquelle la personne travaille.
    \item \textit{mail}, il s'agit de l'adresse de courrier électronique de la personne
\end{itemize}

Voici une liste des principales commandes d'un LDAP :
\begin{itemize}
    \item \textit{abandon} : Abandonne l'opération précédemment envoyée au serveur;
    \item \textit{add} : Ajoute une entrée au répertoire;
    \item \textit{bind} : Initie une nouvelle session sur le serveur LDAP;
    \item \textit{compare} : Compare les entrées d'un répertoire selon des critères;
    \item \textit{delete} : Supprime une entrée d'un répertoire;
    \item \textit{extended} : Effectue des opérations étendues;
    \item \textit{rename} : Modifie le nom d'une entrée;
    \item \textit{search} : Recherche des entrées d'un répertoire;
    \item \textit{unbind} : Termine une session sur le serveur LDAP.
\end{itemize}

  \section{NFS}
    \input{nf.tex}
  \section{Fichiers logs}
    \subsubsection{Définition}

\paragraph{} Dans le domaine informatique, le terme ``log'' désigne un type de
fichier, ou une entité équivalente, dont la mission principale consiste à
stocker un historique des événements. Diminutif de \textit{logging}, le terme
peut être traduit en français par "journal". Le log s'apparente ainsi à un
journal de bord horodaté, qui ordonne les différents événements qui se sont
produits sur un ordinateur, un serveur, etc.

\paragraph{} Ils s'avèrent utiles pour comprendre la provenance d'une erreur en
cas de bug. Ils permettent également d'établir des statistiques, comme le
nombre de connexions à un site Web ou à un serveur, le nombre d'échec
d'authentification, etc.

\paragraph{} S'il s'agit d'un serveur Web, un fichier log va par exemple
enregistrer la date et l'heure de la tentative d'accès, l'adresse IP du client,
le fichier cible, le système d'exploitation utilisé, le navigateur, la réponse
du serveur à cette requête, éventuellement le type d'erreur rencontré...

\paragraph{} Les fichiers de log peuvent contenir des informations
confidentielles

  \section{Virtual Machine}
    \subsubsection{Définitions}

\paragraph{La Virtualisation}
 Au sens large, la virtualisation consiste à simuler l'existence de plusieurs machines informatiques en utilisant une seule.
 Ceci permet en particulier de diminuer les coûts d'achat de matériel informatique et de rentabiliser leur utilisation.
Exemples de logiciels de virtualisation: VMware, VirtualBox...

\paragraph{Un Hyperviseur}
\begin{itemize}
	\item assure le contrôle du processeur et des ressources de la machine hôte
	\item Alloue à chaque machine virtuelle (VM) les ressources dont elle a besoin
	\item S'assure que ces VM n'interfèrent pas l'une avec l'autre
\end{itemize}

Il existe deux types d'hyperviseurs: le type 1 et le type 2

\paragraph{Définition d'une machine virtuelle}

Une machine virtuelle est un fichier informatique, généralement appelé image, qui se comporte comme un ordinateur réel.
En d’autres termes, il s’agit d’un ordinateur créé à l’intérieur d’un ordinateur. Elle s’exécute dans une fenêtre,
 comme tout autre programme, en offrant à l’utilisateur final une expérience identique à celle qu’il aurait sur le
 système d’exploitation hôte. La machine virtuelle est placée dans un « bac à sable » qui l’isole du reste du système,
 de sorte que les logiciels installés sur la machine virtuelle ne peuvent ni s’échapper, ni modifier l’ordinateur hôte.
 Cela produit un environnement idéal pour tester d’autres systèmes d’exploitation, dont des versions bêta,
 l’accès à des données infectées par des virus, la création de sauvegardes de système d’exploitation et l’exécution
 de logiciels ou d’applications sur des systèmes d’exploitation auxquels ils ne sont pas destinés à l’origine.
Il est possible d’exécuter plusieurs machines virtuelles simultanément sur un même ordinateur physique.
Pour les serveurs, les divers systèmes d’exploitation fonctionnent côte à côte, avec un composant logiciel appelé hyperviseur
(logicielle de virtualisation) pour les gérer, alors que les ordinateurs de bureau classiques n’utilisent qu’un seul
système d’exploitation pour exécuter d’autres systèmes d’exploitation dans des fenêtres de programme qui leur sont propres.
 Chaque machine virtuelle fournit son propre matériel virtuel, à savoir les processeurs, la mémoire, les disques durs,
  les interfaces réseau et les autres périphériques nécessaires. Le matériel virtuel est ensuite mappé au matériel réel
   sur la machine physique, ce qui permet de réaliser des économies en réduisant le besoin de disposer de systèmes matériels physiques,
    ainsi que les coûts de maintenance associés, tout en réduisant la demande en alimentation et refroidissement.


VirtualBox est une application de virtualisation x86 qui facilite la création de machines virtuelles mais pas très conseillé
d’utilisation, il est préférable d'utiliser Docker(lxd) qui est très léger et permet d'éviter une surcharge
Sinon nous avons KVM qui lui est un peu plus lourd mais va plutôt se rapprocher de virtual Box.

  \section{Infra du club}
    \input{infra.tex}

\end{document}
